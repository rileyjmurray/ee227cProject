\documentclass[12pt]{article}
\usepackage{fullpage}
\usepackage[]{mathtools}
\usepackage{amsthm}
\DeclareMathOperator{\Tr}{Tr}
\usepackage[]{thmtools}
\usepackage[]{bbm}
\usepackage[]{amssymb}
\usepackage[]{eufrak}
\usepackage{multicol}

%Environments
\newcommand{\be}[1]{\begin{eqnarray}#1\end{eqnarray}}
\newcommand{\al}[1]{\begin{align}#1\end{align}}
\newcommand{\iit}[1]{\begin{itemize}#1\end{itemize}}
\newcommand\bmat[1]{\begin{bmatrix}#1\end{bmatrix}}

%Genuine Shortcuts
\newcommand{\lrr}{\Longleftrightarrow}
\newcommand{\rr}{\Rightarrow}
\newcommand{\fa}{\ \forall \ }

\newcommand{\Rn}[1]{\mathbb{R}^{#1}}
\newcommand{\Pb}{\mathbb{P}}
\newcommand{\Ex}{\mathbb{E}}
\newcommand{\R}{\mathbb{R}}
\newcommand{\N}{\mathcal{N}}
\newcommand{\C}{\mathcal{C}}
\newcommand{\Lm}{\mathcal{L}}
\newcommand{\Sm}{\mathcal{S}}
\newcommand{\Ra}{\mathcal{R}}
\newcommand{\D}{\Delta}
\newcommand{\G}{\Gamma}
\newcommand{\La}{\Lambda}
\newcommand{\la}{\lambda}
\newcommand{\Si}{\Sigma}

\newcommand{\simax}{{\sigma_{\mathrm{max}}}}
\newcommand{\simin}{{\sigma_{\mathrm{min}}}}
\newcommand{\lmax}{{\lambda_{\mathrm{max}}}}
\newcommand{\lmin}{{\lambda_{\mathrm{min}}}}

\newcommand{\prob}{{\mbox{\rm Prob}}}
\newcommand{\var}{{\mbox{\rm var}}}
\newcommand{\sint}{{\mbox{\rm int}\,}} %set interior
\newcommand{\relint}{{\mbox{\rm relint}\,}} %set interior
\newcommand{\ns}{{\mbox{\tt ns}}} 

\newcommand{\lrb}[1]{\left(#1\right)}
\newcommand{\lrbb}[1]{\left[#1\right]}
\newcommand{\inv}{^{-1}}
\newcommand{\trans}{^T}

\newcommand{\os}[1]{\overset{#1}}
\newcommand{\us}[1]{\underset{#1}}
\newcommand{\sgn}[1]{\mbox{sgn}\lrb{#1}}
\newcommand{\tr}[1]{\mbox{Tr}\lrb{#1}}



\begin{document}
\begin{multicols}{2}
\noindent An instance $\mathcal{C}$ of CSP($\Gamma$) is characterized by
\begin{itemize}
\item a set of $n$ variables (denoted $V$), 
\item a set of $m$ constraints (denoted $C$),
\end{itemize}
We write this as a triple $\mathcal{C} = (V,C)$.
\columnbreak
\\
\noindent Each constraint $C_i \in C$ is characterized by
\begin{itemize}
\item a list of variables associated with the constraint (called the \textit{scope} of $C_i$; denoted $S_i$),
\item and a function $R_i \in \Gamma$ that operates on assignments of variables in $S_i$.
\end{itemize}
We write $C_i$ as an ordered pair $C_i = (R_i, S_i)$.
\end{multicols}
If $F$ (a mapping $ V\to D$) is a given assignment of variables, then $R_i(F(S_i))$ equals either 1 or 0, in which case constraint $C_i$ is said to be ``satisfied" or ``not satisfied" respectively.
The objective is to maximize the weighted sum of satisfied constraints $\sum_{i = 1}^m w_i R_i(F(S_i))$. For a given $F$ we call this value as  $\text{Val}_{\mathcal{C}}[F]$.

$\mathcal{C}$ is said to be \textit{satisfiable} if $\forall i, ~ R_i(F(S_i)) = 1$ at optimality.
\newpage
$\Omega_D^k$ \qquad  $D$ \qquad $\mathcal{C}$ \qquad $ \mathfrak{C}$ \qquad $k$ \qquad $\{\neq\}$ \qquad $\{ \cdot \vee \cdot \vee \cdot \}$ \qquad $\Gamma \subset \Omega_D^k$ \qquad ${5 \choose 2}$
\newline
\newline

\begin{alignat}{2}
\max_{\substack{y \geq 0 \\ X \succeq 0 }} ~&~ 
	\sum_{i : C_i\in C} \sum_{L \in \mathcal{L}_i} R_i(L) y_i[L]  & \nonumber \\
s.t. ~ & ~ \sum_{L \in \mathcal{L}_i} y_i[L] = 1  & \forall i : C_i \in C  \\
     ~ & ~~ \sum_{\substack{L\in \mathcal{L}_{i} \\ L(v) = \ell \\ L(v')=\ell'} } y_{i}[L]  = X_{\lrb{v, \ell},\lrb{v', \ell'}}  \qquad &~\forall ~ v,v' \in V, \ell,\ell' \in D, i : C_i \in C \\
     ~& ~ X_{\lrb{v, \ell},\lrb{v, \ell'}} = 0 \quad \forall ~ \ell' \neq \ell & \forall ~ \ell' \neq \ell,  \forall ~ v \in V 
\end{alignat}

\al{
X_{\lrb{v, \ell},\lrb{v', \ell'}} ~ = ~ \Ex\lrbb{I_v(\ell) I_{v'}(\ell')}  ~ = ~ \Pb_{L\sim y_i} \lrbb{L(v) = \ell, L(v')= \ell'} \quad \forall ~ i : C_i \in C
}

\al{
&&X_{\lrb{v, \ell},\lrb{v, \ell'}} &= 0 &\fa \ell \neq \ell' , v \in V\\
&\rr &\Pb_{L \sim y_{i}}\lrbb{L(v) = \ell, L(v)= \ell'} &= 0 &\fa \ell \neq \ell', v \in V, i:C_i \in C
}

\end{document}