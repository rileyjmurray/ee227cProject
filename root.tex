\documentclass[letterpaper, 12pt]{article}

\usepackage[margin=2.5cm]{geometry}
\usepackage{amsmath,amsthm,amssymb}
\usepackage[]{mathtools}
\usepackage[]{bbm}
\usepackage{booktabs,tabularx}
\newcolumntype{C}{>{\arraybackslash}X}
\newcolumntype{s}{>{\arraybackslash\hsize=.5\hsize}X}
\newcolumntype{m}{>{\arraybackslash\hsize=.85\hsize}X}

% listings settings
\usepackage{listings}
\usepackage{color}

\definecolor{mygreen}{rgb}{0,0.6,0}
\definecolor{mygray}{rgb}{0.5,0.5,0.5}
\definecolor{mymauve}{rgb}{0.58,0,0.82}

\lstset{ %
	backgroundcolor=\color{white},   % choose the background color; you must add \usepackage{color} or \usepackage{xcolor}
	basicstyle=\footnotesize,        % the size of the fonts that are used for the code
	breakatwhitespace=false,         % sets if automatic breaks should only happen at whitespace
	breaklines=true,                 % sets automatic line breaking
	captionpos=b,                    % sets the caption-position to bottom
	commentstyle=\color{mygreen},    % comment style
	deletekeywords={...},            % if you want to delete keywords from the given language
	escapeinside={\%*}{*)},          % if you want to add LaTeX within your code
	extendedchars=true,              % lets you use non-ASCII characters; for 8-bits encodings only, does not work with UTF-8
	frame=single,	                   % adds a frame around the code
	keepspaces=true,                 % keeps spaces in text, useful for keeping indentation of code (possibly needs columns=flexible)
	keywordstyle=\color{blue},       % keyword style
	language=Octave,                 % the language of the code
	otherkeywords={*,...},           % if you want to add more keywords to the set
	numbers=left,                    % where to put the line-numbers; possible values are (none, left, right)
	numbersep=5pt,                   % how far the line-numbers are from the code
	numberstyle=\tiny\color{mygray}, % the style that is used for the line-numbers
	rulecolor=\color{black},         % if not set, the frame-color may be changed on line-breaks within not-black text (e.g. comments (green here))
	showspaces=false,                % show spaces everywhere adding particular underscores; it overrides 'showstringspaces'
	showstringspaces=false,          % underline spaces within strings only
	showtabs=false,                  % show tabs within strings adding particular underscores
	stepnumber=5,                    % the step between two line-numbers. If it's 1, each line will be numbered
	stringstyle=\color{mymauve},     % string literal style
	tabsize=2,	                   % sets default tabsize to 2 spaces
	title=\lstname                   % show the filename of files included with \lstinputlisting; also try caption instead of title
}

\numberwithin{equation}{section}

% --to donotes
\usepackage{xargs}                      % Use more than one optional parameter in a new commands
\usepackage[dvipsnames]{xcolor}
\usepackage[colorinlistoftodos,prependcaption,textsize=tiny]{todonotes}
\newcommandx{\unsure}[2][1=]{\todo[linecolor=red,backgroundcolor=red!25,bordercolor=red,#1]{#2}}
\newcommandx{\change}[2][1=]{\todo[linecolor=blue,backgroundcolor=blue!25,bordercolor=blue,#1]{#2}}
\newcommandx{\info}[2][1=]{\todo[linecolor=OliveGreen,backgroundcolor=OliveGreen!25,bordercolor=OliveGreen,#1]{#2}}
\newcommandx{\maybeinclude}[2][1=]{\todo[linecolor=Orange,backgroundcolor=Orange!25,bordercolor=Orange,#1]{#2}}
\newcommandx{\improvement}[2][1=]{\todo[linecolor=Plum,backgroundcolor=Plum!25,bordercolor=Plum,#1]{#2}}

\usepackage[]{thmtools}
\usepackage[dvipsnames]{xcolor}
\declaretheoremstyle[
	bodyfont=\normalfont, 
	spaceabove=0.5cm]{defFormat}
\declaretheoremstyle[
	postheadspace=1cm,
	bodyfont=\normalfont, 
	spaceabove=0.5cm]{inLineDefFormat}
\declaretheoremstyle[
	spaceabove=0.5cm, 
	spacebelow=0.5cm,
	postheadspace=1cm]{namedTheorem}
\declaretheorem[
	numberwithin=section, 
	style=defFormat, 
	shaded]{definition}
\declaretheorem[
	numbered=no, 
	style=defFormat, 
	shaded]{algorithm}
\declaretheorem[
	style=inLineDefFormat, 
	sibling=definition,
	shaded,
	name=Definition]{ILdefinition}
\declaretheorem[
	numbered=no, 
	style=namedTheorem, 
	shaded, 
	name=The Unique Games Conjecture]{ugc}
\newtheorem{thm}{Theorem}
\declaretheorem[numberwithin=section, style=defFormat, shaded]{Definition}

% Shortcuts:

%Theorems
\newtheorem{theorem}{Theorem}
\newtheorem{lemma}{Lemma}
\newtheorem{prop}{Proposition}
\newtheorem{remark}{Remark}
\newtheorem{example}{Example}
%
%Environments
\newcommand{\be}[1]{\begin{eqnarray}#1\end{eqnarray}}
\newcommand{\al}[1]{\begin{align}#1\end{align}}
\newcommand{\iit}[1]{\begin{itemize}#1\end{itemize}}
\newcommand\bmat[1]{\begin{bmatrix}#1\end{bmatrix}}

%Genuine Shortcuts
\newcommand{\lrr}{\Longleftrightarrow}
\newcommand{\rr}{\Rightarrow}
\newcommand{\fa}{\ \forall \ }

\newcommand{\Rn}[1]{\mathbb{R}^{#1}}
\newcommand{\Pb}{\mathbb{P}}
\newcommand{\Ex}{\mathbb{E}}
\newcommand{\R}{\mathbb{R}}
\newcommand{\N}{\mathcal{N}}
\newcommand{\C}{\mathcal{C}}
\newcommand{\Sm}{\mathcal{S}}
\newcommand{\Ra}{\mathcal{R}}
\newcommand{\D}{\Delta}
\newcommand{\G}{\Gamma}
\newcommand{\La}{\Lambda}
\newcommand{\la}{\lambda}
\newcommand{\Si}{\Sigma}

\newcommand{\simax}{{\sigma_{\mathrm{max}}}}
\newcommand{\simin}{{\sigma_{\mathrm{min}}}}
\newcommand{\lmax}{{\lambda_{\mathrm{max}}}}
\newcommand{\lmin}{{\lambda_{\mathrm{min}}}}

\newcommand{\prob}{{\mbox{\rm Prob}}}
\newcommand{\var}{{\mbox{\rm var}}}
\newcommand{\sint}{{\mbox{\rm int}\,}} %set interior
\newcommand{\relint}{{\mbox{\rm relint}\,}} %set interior
\newcommand{\ns}{{\mbox{\tt ns}}} 

\newcommand{\lrb}[1]{\left(#1\right)}
\newcommand{\lrbb}[1]{\left[#1\right]}
\newcommand{\inv}{^{-1}}
\newcommand{\trans}{^T}

\newcommand{\os}[1]{\overset{#1}}
\newcommand{\us}[1]{\underset{#1}}
\newcommand{\sgn}[1]{\mbox{sgn}\lrb{#1}}
\newcommand{\tr}[1]{\mbox{Tr}\lrb{#1}}


\DeclareMathOperator{\grad}{\nabla}  % gradient

\newcommand{\Real}[1]{ { {\mathbb R}^{#1} } }

\renewcommand{\vec}[1]{\mathbf{#1}}
\newcommand{\norm}[1]{\lVert #1 \rVert}
\DeclareMathOperator{\Tr}{Tr}

\parindent 0pt
\parskip 8pt

\begin{document}
	% CSP
	\begin{titlepage}
	
	\begin{center}
		\vspace{10cm}
		
		% Upper part of the page
		\includegraphics[width=.5\textwidth]{./Images/UCBlogo}\\[3cm]    
		
		\normalsize EE 227B\\
		\textsc{\large Convex Optimization}\\[1cm]		
		
		% Title
		\hrule 
		\vspace{1 cm}
		{ \Large \textbf{Convex Relaxation for Constraint Satisfaction Problems}\\[0.5cm]
			\vspace{0.5 cm}
			\hrule
			\vspace{1.5 cm}
			
			% Author and supervisor
			\begin{minipage}[t]{0.4\textwidth}
				\begin{flushleft} \large
					\emph{Authors:}\\
					\vspace{0.7ex}
					Raaz \textsc{Dwivedi}\\
					Riley \textsc{Murray}\\
					Quico \textsc{Spaen}
				\end{flushleft}
			\end{minipage}
			\begin{minipage}[t]{0.4\textwidth}
				\begin{flushright} \large
					\emph{Instructor:} \\
					\vspace{0.7ex}
					Laurent \textsc{El Ghaoui}\\[0.3 cm]
				\end{flushright}
			\end{minipage}
			\vfill 
			% Bottom of the page
			University of California, Berkeley\\[.5cm]
			\large \today}
		
	\end{center}
	
\end{titlepage}


	% Reader introduction
	\section{Words go here}
and here
	\newpage	
	% prelim
	\section{Preliminaries}
The study of Constraint Satisfaction Problems is of significant interest precisely because many practical problems are intractable.  

It is assumed that the reader has some familiarity fundamentals of algorithmic complexity, including big-Oh notation for algorithm runtime, as well as NP-hardness and NP-completeness. 
For the reader's convenience, we define some key terms in complexity theory below.

A reader with some knowledge of ``approximation algorithms" is likely to better appreciate the material in this report, but strictly speaking, the prior knowledge is not necessary. 
We provide a very brief introduction to approximation algorithms below. 
Those experienced in the field should pay special attention to Definition \ref{def:twoParamApproxAlg}, which has seen only limited use.

The focus of this report will be the use of Semidefinite Programming for CSP's. As such, the reader is expected to be proficient in linear algebra and to have some exposure to mathematical programming (linear programming at minimum).

\subsection{NP-Hardness and NP-Completeness}
\begin{ILdefinition}
A \textbf{decision problem} is an algorithmic question.
Given some input data, and a set of rules, does the input data satisfy the rules?
\end{ILdefinition}
It is implied that sufficient data is provided to definitively answer ``yes" or ``no", even if determining the answer might take a prohibitively long amount of time.
\begin{ILdefinition}
\textbf{NP} is a collection of decision problems. A decision problem is in \textbf{NP} if for every ``yes" answer, there is an efficient procedure to verify that the answer is in fact ``yes." 
There are no requirements for ``no" answers.
\end{ILdefinition}
\begin{ILdefinition}
A decision problem $\mathcal{P}$ is said to be \textbf{NP-Complete} if (1)
$\mathcal{P} \in $ NP, and (2) any other problem $\mathcal{Q} \in$ NP can be stated ``succinctly" in terms of $\mathcal{P}$ .
\end{ILdefinition}
By ``succinctly", we mean that the transformation from $\mathcal{Q}$ to the equivalent problem in the terms of $\mathcal{P}$ can be done in both polynomial time and space.\footnote{This process of carrying out this transformation is usually called a ``reduction."}
\begin{ILdefinition}
A problem $\mathcal{P}$ (which may or may not be a decision problem) is said to be \textbf{NP-Hard} if every problem in NP can be stated succinctly in terms of $\mathcal{P}$.
\end{ILdefinition}

\subsection{Approximation Algorithms for Optimization Problems}\label{subsec:intractProbCope}

Although ``problems" are said to be intractable, we actually only solve problem \textit{instances}. 
In a variety of circumstances, it is reasonable to solve an instance of an intractable problem with an exact but exponential algorithm.

For example, there is billions of dollars of capital involved in coordinating the movements of even a handful of trans-oceanic shipping vessels. 
Although not necessarily the case, it is very likely that such a logistics problem would be NP-Hard.
Nevertheless, if there are a sufficiently small number of decisions to be made in this planning process, it could make perfect sense to solve the planning problem with an exact algorithm.

The situation changes slightly when deadlines are involved, since getting \textit{some} solution by a deadline is often more important than getting the \textit{best} solution after that time. 
The presence of imminent deadlines does not completely rule out the use of exact algorithms; high powered computers with sophisticated (but still exponential) algorithms are often used under these circumstances. 
Logistics for airlines is one prominent example.

But when an intractable problem has thousands of variables, exact methods are typically worthless. 
For those facing intractable problems of this scale, algorithms which provide solutions within a reasonable amount of time are of paramount importance.  
When the algorithm \textit{does} have a performance guarantee, it is referred to as an ``approximation algorithm."
We give a definition below in the case where the objective is to maximize some function. 

\begin{definition}
\textbf{($\alpha$)-Approximation Algorithm} \\
Let $\Omega$ denote the set of all possible instances of a given maximization problem. 
Let $A$ denote an efficient algorithm which returns a feasible but potentially sub-optimal solution for any $I \in \Omega$. 
Denote the value of the solution returned by $A$ on $I$ as $A(I)$, and denote the value of the optimal solution for $I$ by $OPT(I)$. We call $A$ an $\alpha$-approximation if
\begin{equation*}
\frac{OPT(I)}{A(I)} \leq \alpha ~ \forall I \in \Omega
\end{equation*}
\label{def:commonApproxAlg}
\end{definition}

Definition \ref{def:commonApproxAlg} is the most common definition of an approximation algorithm, and is suitable for many applications. It can be useful, however, to describe how performance guarantees relate to the optimal objective value. 
\begin{definition}
\textbf{$(\alpha,\beta)$-Approximation Algorithm } \\
Let $\Omega$, $I$, and $A$ be as before. 
Define $\Omega_\beta$ as the set of all problem instances with $OPT(I) \geq \beta \forall I \in \Omega_\beta$.
We call $A$ an $(\alpha,\beta)$ approximation if 
\begin{equation*}
\frac{OPT(I)}{A(I)} \leq \alpha ~ \forall I \in \Omega_{\beta}
\end{equation*}
\label{def:twoParamApproxAlg}
\end{definition}
\newpage
	\newpage
	% CSP introduction
	%\documentclass[12pt]{article}
%\usepackage{fullpage}
%\usepackage[]{mathtools}
%\usepackage{amsthm}
%\DeclareMathOperator{\Tr}{Tr}
%\usepackage[]{thmtools}
%\usepackage[dvipsnames]{xcolor}
%\declaretheorem{theorem}
%\declaretheoremstyle[
%	bodyfont=\normalfont, 
%	spaceabove=0.5cm]{defFormat}
%\declaretheoremstyle[
%	postheadspace=1cm,
%	bodyfont=\normalfont, 
%	spaceabove=0.5cm]{inLineDefFormat}
%\declaretheoremstyle[
%	spaceabove=0.5cm, 
%	spacebelow=0.5cm,
%	postheadspace=1cm]{namedTheorem}
%\declaretheorem[
%	numberwithin=section, 
%	style=defFormat, 
%	shaded]{definition}
%\declaretheorem[
%	style=inLineDefFormat, 
%	sibling=definition,
%	shaded,
%	name=Definition]{ILdefinition}
%\declaretheorem[
%	numbered=no, 
%	style=namedTheorem, 
%	shaded, 
%	name=The Unique Games Conjecture]{ugc}
%\declaretheorem[numbered=no, style=remark]{remark}
%\usepackage[]{bbm}
%\usepackage[]{amssymb}
%\begin{document}

%\parindent 0pt
%\parskip 8pt

\section{Introduction to Constraint Satisfaction Problems}\label{sec:introToCSP}
There are conflicting conventions regarding what is and is not a CSP. This section is devoted to defining ``CSP"'s in the way that is usually meant by the theoretical computer science community. Some alternative characterizations of CSP's are addressed at the end of this section. 
\subsection{Getting Oriented with CSP's}

There is a great deal of research going into the approximabilty of constraint satisfaction problems, and some claims regarding CSP's can seem quite sensational. 
The list of points below gives some facts relating to CSP's to help orient the reader.

\begin{itemize}
\item At their core, CSP's are \textit{optimization} problems. But as we will see, CSP's can be used to approach \textit{decision} problems. The reader is advised that some literature on CSP's conflate these two problem types.
\item The objective of a CSP is to satisfy as many constraints as possible; there is only one constraint that is ``safe" from violation: all variables must be in some simple, discrete domain.
\item Most CSP's are NP-Hard to solve optimally. Because of this, discussion of CSP's centers on developing approximate solutions to these problems.
\item Semidefinite Programming is the primary technique for CSP approximation.
\item The ability to generate a near optimal-solution for a CSP is intimately related to an open problem in computer science known as the ``Unique Games Conjecture."
\end{itemize}

\subsection{The CSP Framework and CSP Instances}\maybeinclude{A section dedicated for formulating a variety of problems as CSPs? The reader will care less and less if all we talk about is max cut and max SAT.}
Define the \textit{arity} of an indicator function $R$ as the number of arguments it takes, and denote the arity by $\text{ar}(R)$.

\begin{definition}
\textbf{The CSP Framework} \\
Let $D$ be a finite domain of fixed cardinality $q$. 
Let $R$ denote an indicator function over $D$ with arity $r \leq k$ (i.e. $R:D^r \to \{0,1\}$). 
Let $\Gamma$ be a possibly exhaustive set of such functions. 
$D$ and $\Gamma$ define a \textit{class} of problems which we denote CSP($\Gamma$).
\label{def:CSPframework}
\end{definition}

We usually write $D = \{0,1,\ldots,q-1\}$, although the elements of $D$ can serve as \textit{labels}, without any of the algebreic structure implied by the use of integers.

\textbf{Examples}
\begin{itemize}
\item Max-Cut : For a graph $G = (V,E)$, label each vertex either ``$0$" or ``$1$" to maximize the number of edges connecting vertices with different labels.   
Here $D = \{0,1\}$, and all constraints are of the form $\{v_i \neq v_j\}$. We could write $\Gamma = \{\neq\}$ as the ``not equal" operator. 
\item $q-$Coloring : Find an as-consistent-as-possible coloring for a graph using at most $q$ colors. 
Our domain is $D = \{0,1,\ldots,q-1\}$, and $\Gamma$ is again $\{\neq\}$. 
Note that Max-Cut is equivalent to $q-$coloring with $q = 2$.
\item Max E3-SAT : Here, the domain is $D = \{0,1\}$, and our constraint types (elements of $\Gamma$) are of the form $(x_j \vee x_j \vee x_k)$, $(x_i\vee \bar{x}_j \vee x_k)$, $(\bar{x}_i \vee \bar{x}_j \vee x_k)$, $\ldots$ - all disjunctions on \textit{`E'xactly} $3$ literals. 
\item Max $k-$SAT : $D = \{0,1\}$, $\Gamma$ is all disjunctions of \textit{up to} $k$ literals.
\item $3-$CSP : $D = \{0,1\}$, $\Gamma$ is all relations on up to three binary variables.
\end{itemize}

\begin{definition}
\textbf{CSP Instance}\\
An instance $\mathcal{C}$ of CSP($\Gamma$) is characterized by a set of $n$ variables (denoted $V$), a set of $m$ constraints (denoted $C$), and $m$ positive weights (one for each constraint). We write this as a triple $\mathcal{C} = (V,C,W)$.
Every constraint $C_i \in C$ has the form $C_i = (R_i,S_i)$ where $R_i \in \Gamma$ and $S_i$ (said to be the \textit{scope} of $C_i$) is a possibly ordered list of \text{ar}($R_i$) variables. 

If $F$ (a mapping $ V\to D$) is a given assignment of variables, then $R_i(F(S_i))$ equals either 1 or 0, in which case constraint $C_i$ is said to be ``satisfied" or ``not satisfied" respectively.
The objective is to maximize the weighted sum of satisfied constraints $\sum_{i = 1}^m w_i R_i(F(S_i))$. For a given $F$ we call this value as  $\text{Val}_{\mathcal{C}}[F]$.

$\mathcal{C}$ is said to be \textit{satisfiable} if $\forall i, ~ R_i(F(S_i)) = 1$ at optimality.
\label{def:CSPinstance}
\end{definition}

\subsubsection{Discussion of Definition \ref{def:CSPinstance}}\maybeinclude{Generalized CSP's with [-1,1] payoffs. Could then include ``MultiwaiCut" and "MetricLabeling."}
\textbf{Remarks on the objective function - }Without loss of generality, we may take the weights to sum to 1. 
When we do this, we can interpret the weights as probabilities and introduce a random variable $\tilde{C}$  which takes values in the set of constraints, i.e., it takes values in $C$. In particular, $\tilde{C} = C_i$ with probabiilty $w_i$. We denote this as $ \tilde{C} \sim W$, where $W$ stands for the probability mass function for the random variable $\tilde{C}$. This random variable view comes in handy later on. Next we define the random variable $R_F(\tilde{C}) = R_i(F(S_i))$ with probability $w_i$. And thus in this notation, we can write our objective as to maximize the expectation as expressed below -
\begin{equation} \label{expfirst}
\max_{F}  \text{Val}_{\mathcal{C}}[F] = \max_F \mathbb{E}_{\tilde{C} \sim W}\left[R_F(\tilde{C})\right] = \sum_{i=1}^m w_i R_i(F(S_i))
\end{equation}


\textbf{Remarks on the constraints - }In almost all optimization paradigms, ``constraints" are by their very definition \textit{inviolable}. 
Definition \ref{def:CSPinstance} departs from this convention in that a feasible solution is not required to satisfy any of a CSP's ``constraints." 
Are CSP's then accurately understood as unconstrained optimization? 
Not quite. 
The CSP Framework has one and only one inviolable constraint (and that constraint is precisely what makes them difficult): each of the $n$ variables $x_i, ~ i \in\{1,\ldots,n\}$ belongs to the discrete domain $D$. 

\subsection{Complexity for Solving and Approximating CSP's}
One of the most surprising results in CSP's and approximation algorithms is the lack of a consistent relationship between the difficulty of solving a problem optimally, and the difficulty of approximating the problem.

Take 3-SAT for example. 
It is well known that it is NP-Hard to find a satisfiable assignment to a 3-SAT instance even \textit{given} that it is satisfiable. 
It is equivalent to say that it is NP-Hard to $(1,1)$ approximate ``Max 3-SAT." 
In spite of this, Max 3-SAT has a $(7/8,1)$ polynomial time approximation algorithm \cite{Zwick1997}, and a $(0.784\beta,\beta)$ approximation algorithm \cite{AsanoAndWilliamson2002}.\maybeinclude{There is also an approximation algorithm for MAX E3-SAT at(7/8$\beta$,$\beta$). It is in fact NP-hard do to better than (7/8$\beta$,$\beta$) for Max E3-SAT and Max 3-SAT.}
From this we might expect that approximating an optimization problem (Max 3-SAT) is easier than solving the corresponding decision problem (3-SAT). 
Alas, this is not always the case. 

For example, the 2-SAT decision problem can be solved in polynomial time, but Max 2-SAT is NP-Hard to approximate by better than 0.955. \cite{Hastad2001}
The Max-Cut problem is yet another example of this phenomenon. 
There is a polynomial time algorithm for determining whether or not a Max-Cut instance is satisfyable\footnote{one need only check whether the graph is bipartite}, 
but the best known approximation algorithm for Max-Cut in the general case is a $(0.878\beta,\beta)$ approximation \cite{gwFirstMaxCutSDP}. 
It has been proven that it is NP-Hard to approximate Max-Cut by a factor better than 0.942 \cite{Trevisan2000}, 
but it may yet be NP-Hard to do better than 0.878! 

Whether existing approximation algorithms can be improved for a huge swath of combinatorial problems (including Max-Cut), depends heavily on the truth of the Unique Games Conjecture. 
We address this next.

\subsection{Hardness of Approximation: The Unique Games Conjecture} 
In 2002, Subhash Khot published a paper entitled \textit{On the Power of Unique 2-Prover 1-Round Games} \cite{Khot}. 
In his paper, Khot put forward the Unique Games Conjecture- currently one of the most important open problems in theoretical computer science.

It can be difficult to get a handle on what the Unique Games Conjecture claims. 
The purpose of this section is to provide the reader with the background necessary to understand how the Unique Games Conjecture relates to CSP approximation. 
This will become critically important in subsequent sections on SDP relaxations of CSP's.

\begin{itemize}
\item The UGC is a statement of the hardness of approximation: 
for some problems, it is NP-Hard to determine whether every solution for a given instance is \textit{extremely poor}, or whether an \textit{almost perfect} solution exists for the instance.
\item If true, the UGC would imply that many already existing approximation algorithms cannot be improved upon. 
This includes approximation algorithms which are not for CSP's and do not use semidefinite programming.
\end{itemize}

\subsubsection{What is a ``Unique Game"?}

A ``unique game" is a game in the game-theory sense that relates to Probabilistically Checkable Proofs (PCP's). 
It is a slightly less general version of ``2-Prover 1-Round games." 
Below we list the three conditions of 2-Prover 1-Round games, and then the additional condition that makes such a game ``unique."
\begin{itemize}
\item The game pits two \textit{provers} against a \textit{verifier}.
\item The verifier asks one question of each prover.\footnote{Questions are drawn from appropriate probability distributions.}
\item Given the answers returned by the two provers, the verifier returns ``True" or ``False."
\item[\textbf{*}] The answer of one prover \textit{uniquely} determines the answer of the other prover.
\end{itemize}

\subsubsection{What does UGC have to do with CSP's or combinatorial optimization?}

Khot posed the ``game" in the UGC in three equivalent ways
\begin{itemize}
\item The 2-Prover 1-Round game with uniqueness (specified above).
\item A CSP where the constraints to be satisfied are a system of linear equations in two variables, modulo 2.
\item A new graph-theoretic CSP called ``Label Cover."\footnote{Khot's game chose a particular type of Label Cover problem where the graph is bipartite, but subsequent discussions of Unique Games do not emphasize this characteristic. }
\end{itemize}

If it is claimed that an algorithm is for ``Unique Games," the reader would do well to clarify which formation the author's work with. Because graph theory has a larger research community than those working with Probabilistically Checkable Proofs, several researchers (including Khot himself) primarily use the Label Cover formulation of Unique Games in discourse. 

The interested reader is strongly recommended to read Khot's original paper for a precise definitions for Proababilistically Checkable Proofs, the Two-Prover One-Round ``unique" game, and the Label Cover problem. But suffice it to say

\begin{ugc}
For arbitrarily small constants $\epsilon$, $\delta > 0$, there exists a constant $k = k(\epsilon,\delta)$ such that it is NP-Hard to determine whether a unique Label Cover instance with label sets of size $k$ at satisfies \textit{at least} $1-\epsilon$ or \textit{at most} $\delta$ constraints.
\end{ugc}

\subsection{Conflicting Conventions in CSP Discourse}

\subsubsection{``Hard" or ``Soft" Constraints?}
Lots of stuff\cite{Russell2009}.
MIT slides.

\subsubsection{``Constraints", ``Payoffs", and Generalized Constraint Satisfaction Problems}

The reader is advised that some papers (most notably \cite{RagSte09}) use the term ``payoff" for some $C_i = (R_i,S_i) \in \mathcal{C}$. 

Why do this? Reasons!

Besides the obvious usefulness of removing confusion of whether or not CSP constraints are violable, a payoff function is used in the context of \textit{Generalized} Constraint Satisfaction Problems.

Examples of GCSP's
\begin{itemize}
\item Multiway Cut
\item Metric Labeling
\end{itemize}

\subsection{Domains : Finite, or Uncountably Infinite?}
Lots of stuff\cite{Russell2009}.

%\end{document}

	\newpage
	% lp relaxation
	\section{Linear Programming for CSP Approximation}\label{sec:lpRelax}
The goal of this section is to present a ``canonical" LP relaxation and rounding scheme for Constraint Satisfaction Problems. Both the relaxation and rounding scheme are valid for arbitrary CSP's, although performance guarantees are only proven in limited cases.  A non-exhaustive list of CSP-specific LP relaxations is included at the end of this section.

\subsection{An Integer Program : Towards the Canonical LP [CITE]}\label{subsec:ip}
Let $\mathcal{C} = (V,C,W)$ be a CSP over a domain $D$ of size $q$.

Earlier, we defined an \textit{assignment of variables} as a function $F : V \to D$. Critically, the domain of $F$ is the entire set $V$. When this is the case, we could call $F$ a \textit{full assignment}. It is reasonable (and as we will see, helpful!) to consider $F$ as being built from many \textit{local assignments} $L : S \to D$ where $S \subset V$. For a constraint $C_i = (R_i,S_i) \in \mathcal{C}$, we will be interested in the local assignment $L : S_i \to D$. Where before we could write $R_i(F(S_i))$ as the value of the constraint under an assignment, we write $R_i(L)$ when it is given that $L$ is a local assignment for constraint $C_i$.

Now consider an \textit{Integer}-Linear Program over the following variables:
\begin{itemize}
\item $\mu_v[\ell] \in \{0,1\}$ is an indicator that variable $v \in V$ takes value $\ell \in D$
\item $\lambda_i[L] \in  \{0,1\}$ is an indicator that \textit{local assignment} $L$ is used for constraint $C_i$
\end{itemize}
From these definitions, it is clear that for fixed $v$, we need one and only one $\mu_v[\ell]$ to be equal to 1. Encode this constraint as 
\begin{equation}\label{musum}
\sum_{\ell \in D} \mu_v[\ell] = 1.
\end{equation}

Now we define $\mathcal{L}_i$ as the set of all possible local assignments for the variables in constraint $C_i$'s scope. We note that the size of $\mathcal{L}_i$ is exponential in $t = ar(R_i)$ (in fact, it's exactly $|\mathcal{L}_i| = q^t$). This is one of the key reasons why maximum arity is a \textit{fixed parameter} for all $\mathcal{C} \in \text{CSP}(\Gamma)$.

Since any two local assignments $L_1$, $L_2$ in $ \mathcal{L}_i$ are mutually exclusive, we likewise need one and only one of $\lambda_i[L]$ equal to 1 for fixed $i$.
\begin{equation}\label{lambsum}
\sum_{L \in \mathcal{L}_i} \lambda_i[L] = 1 
\end{equation}\label{mulambcons}
To be consistent across $\lambda$ and $\mu$, we need one more constraint.
\begin{equation}
\mu_v[\ell] = \sum_{\substack{ L \in \mathcal{L}_i \\ L(v) = \ell }} \lambda_i[L] 
\end{equation}
Now we need to come up with an objective that mimics the one defined in Equation \ref{expfirst}. We re-write the expression for objective for clarity: 
$$\max_F \sum_{i:C_i \in C} w_i R_i(F(S_i))$$
As discussed in the beginning of this section it is easy to see 
\begin{equation}\label{replacewithlocal}
R_i(F(S_i)) = \sum_{L\in \mathcal L_i}\lambda_i[L] R_i(L).
\end{equation} 
It follows from the facts that whether or not $R_i$ is satisfied, depends only on the local assignment and RHS of \ref{replacewithlocal} is precisely these terms summed over all possible local assignments with multiplicative factor of indicators for the respective assignments. And thus our objective becomes 
\begin{equation}\label{glbobj}
\max_{\mu, \lambda} \sum_{i : C_i \in C} \sum_{L \in \mathcal{L}_i}   w_i\lambda_i[L] R_i(L).
\end{equation}

We get the corresponding LP simply by relaxing $\mu_v[\ell] \in \{0,1\}$ to $\mu_v[\ell] \in [0,1]$ and $\lambda_i[L] \in \{0,1\}$ to $\lambda_i[L] \in [0,1]$.


\subsection{The Canonical Linear Program}

We define the linear program below, then address its probabilistic interpretation and equivalent representations.
\begin{definition}\textbf{Basic LP} \\
Let $\mathcal{C} = (V,C,W)$ be a CSP over domain $D$. The Basic LP for $\mathcal{C}$ is
\begin{alignat}{2}
\max_{\mu, \lambda} ~&~ \sum_{i : C_i \in C} \sum_{L \in \mathcal{L}_i}   w_i\lambda_i[L] R_i(L) & \\
s.t. ~ & ~ \sum_{\ell \in D} \mu_v[\ell] = 1 & \forall v \in V  \label{eq:canonLPmuSum} \\
     ~ & ~ \sum_{L \in \mathcal{L}_i} \lambda_i[L] = 1  & \forall i : C_i \in C \label{eq:canonLPlambdaSum} \\
     ~ & ~ \sum_{\substack{ L \in \mathcal{L}_i \\ L(v) = \ell }} \lambda_i[L] = \mu_v[\ell]  & \forall v \in V, \ell \in D, i : C_i \in C \label{eq:canonLPConsistency} \\
     ~ & ~ 0 \leq \mu_v[\ell] \leq 1 & v \in V, \ell \in D \label{eq:canonLPmuNonNeg}\\
     ~ & ~ 0 \leq \lambda_i[L] \leq 1  & \forall  i : C_i \in C, L \in \mathcal{L}_i  \label{eq:canonLPlambdaNonNeg} 
\end{alignat}
\end{definition}

\begin{lemma}\label{le:super}
For the Basic LP, constraints \ref{eq:canonLPmuSum} and \ref{eq:canonLPmuNonNeg} can be dropped without altering optimal $\lambda^*$ or the optimal objective.
\end{lemma}
\begin{proof}
It is not difficult to see that constraints \ref{eq:canonLPmuSum} and \ref{eq:canonLPmuNonNeg} are implied by \ref{eq:canonLPlambdaSum}, \ref{eq:canonLPConsistency} and \ref{eq:canonLPlambdaNonNeg} :
\begin{align*}
\ref{eq:canonLPlambdaNonNeg} \rm{\ and \ } \ref{eq:canonLPlambdaSum} &\Rightarrow 0\leq  \sum_{\substack{ L \in \mathcal{L}_i \\ L(v) = \ell }} \lambda_i[L] \leq  \sum_{L \in \mathcal{L}_i} \lambda_i[L] = 1  \\
{\rm \ Combine\ with \ }\ref{eq:canonLPConsistency}&\Rightarrow  0\leq \mu_v[L] \leq 1
\end{align*}
Thus, \ref{eq:canonLPmuNonNeg} is implied. 

Also, summing on both sides of \ref{eq:canonLPConsistency} with respect to $\ell \in D$ we get, 
\begin{align*}
{\rm RHS} = \sum_{\ell \in D} \mu_v[\ell] = {\rm LHS} = \sum_{\ell \in D} \sum_{\substack{ L \in \mathcal{L}_i \\ L(v) = \ell }} \lambda_i[L] = \sum_{L \in \mathcal{L}_i} \lambda_i[L] \overset{\ref{eq:canonLPlambdaSum}}=1
\end{align*}
and thus \ref{eq:canonLPmuSum} is implied as well.
\end{proof}

In other words, the remark above renders $\mu_v[\cdot]$ indicators as superfluous. \emph{Nevertheless, we keep them for the discussion that follows, as they help in gaining some insight, help us come up with an intuitive rounding scheme and also lead us to extending the relaxation in a very natural fashion.}

\begin{thm}
	Basic LP is a relaxation for the problem CSP$(\Gamma)$.
\end{thm}
\begin{proof}
	Given an optimal solution for an instance of the problem CSP($\Gamma$), assign $\mu_v[t]$ to be $1$ if the variable $x_v$ takes value $t$ in the optimal solution to the CSP($\Gamma$) instance and $0$ otherwise.
	Furthermore, for each $C_i \in \mathcal{C}$ set $\lambda_{C_i}[y]$ equal to $1$ if $y$ is the local assignment for the scope $S_i$ corresponding to the optimal solution to the CSP and $0$ otherwise. 
	It follows that this is a feasible LP solution. 
	Hence, $OPT(\mathcal{C}) \le OPT_{LP}(\mathcal{C})$.
\end{proof}

\begin{remark}
It is worthwhile to note that conditions in the Basic LP do not ensure that two constraints having same scope have same values for the relaxed indicators $\lambda$ over different local assignments. This constraint can be imposed in the following manner.\\
For $C_{i}=(R_i, S_i)$ and $C_j=(R_j, S_j)$, if we have $S_i = S_j$, then
\begin{equation}\label{extracons}
\lambda_i[L] =   \lambda_j[L] ~ \ \forall L \in \mathcal L_i = \mathcal L_j
\end{equation}
However we ignore this condition owing to the difficulty that pops up in using them in a meaningful way while rounding up the fractional solution (to get a full assignment for the original CSP) that we get from the LP relaxation.  
\end{remark}

\subsection{LP as an Expectation Maximization}
Note that between constraints \ref{eq:canonLPmuNonNeg} and \ref{eq:canonLPmuSum}, a solution $\mu^*_v[\ell]$ to the above LP defines a probability distribution (over $\ell \in D$) of assignments for $v$. That is, if we wanted to randomly generate an assignment $\hat{\ell}_v$ for variable $v$, then we could simply state $\hat{\ell}_v = \ell$ with probability $\mu^*_v[\ell]$. Given this interpretation, we write
\begin{equation}
\mu_v[\ell] = \mathbb{P}\left( \hat{\ell}_v = \ell \right) \quad \text{ where } \quad \hat{\ell}_v \sim \mu_v 
\end{equation}

A similar interpretation holds for $\lambda$; between constraints \ref{eq:canonLPlambdaNonNeg} and \ref{eq:canonLPlambdaSum}, a solution $\lambda^*_i[L]$ defines a probability distribution (over $L \in \mathcal{L}_i$) of possible local assignments for constraint $C_i$. That is, if we wanted to randomly select a local  assignment $\hat{L}_i$ for constraint $C_i$, we could state $\hat{L}_i = L$ with probability $\lambda^*_i[L]$. Given this interpretation, we write
\begin{equation}
\lambda_i[L] = \mathbb{P}\left( \hat{L}_i = L \right) \quad \text{ where } \quad \hat{L}_i \sim \lambda_i 
\end{equation}
  
These distributions are tied together by constraint \ref{eq:canonLPConsistency}. In the probabilistic terms established above, constraint \ref{eq:canonLPConsistency} reads
\begin{equation}
\mathbb{P}\left( \hat{\ell}_v = \ell \right) = \sum_{\substack{L \in \mathcal{L}_i \\ L(v) = \ell}} \mathbb{P}\left( \hat{L}_i = L \right) \qquad \forall v \in V, \ell \in D, i : C_i \in C
\end{equation}
Since the events  $\{\hat{L}_i = L_1 \}$ and $\{\hat{L}_i = L_2 \}$ are mutually exclusive, the right hand side can be rewritten to give the following. This simply states that the probability that a variable $v$ takes on value $\ell$ is the same whether you consider the distribution as being defined from $\mu_v$ or $\lambda_i$, for any $i : C_i \in C$. We refer to this constraint as a \textit{first moment consistency constraint}.
\begin{equation} 
\mathbb{P}\left( \hat{\ell}_v = \ell \right) = \mathbb{P}\left( \bigcup_{L : L(v) =  \ell} \left\{\hat{L}_i = L\right\} \right) \qquad \forall v \in V, \ell \in D, i : C_i \in C
\end{equation}

As we remarked in Section \ref{sec:introToCSP} and Equation \ref{expfirst}, we can write the objective function of a CSP in the probabilistic terms $\max_{F} \mathbb{E}_{\tilde{C} \sim W}\left[R_F(\tilde{C})\right] $. With slight modifications (to reflect the fact that our decision variables are now \textit{local} rather than global assignments), we can make a similar statement. 

We define a new random variable $R(\tilde{C}, L(\tilde{C}))$, where $\tilde{C}$ is as defined before i.e., 
$$\mathbb{P} (\tilde{C} = C_i ) = w_i$$
and 
$$\mathbb{P}(L(\tilde{C}) = L |\tilde{C}=C_i ) =  \lambda_i[L]\  \forall L \in \mathcal{L}_i,$$ 
then we can express our objective in \ref{glbobj} as 
\begin{equation}
\max_\lambda \mathbb{E}_{\tilde{C}, L(\tilde{C})} R(\tilde{C}, L(\tilde{C}))
\end{equation}
Using the formula for conditional expectation, we can write this as 
\begin{equation}
\max_\lambda \mathbb{E}_{\tilde{C}} \left[ \mathbb{E} \left(R(\tilde{C}, L(\tilde{C}))\displaystyle|\tilde{C}\right). \right] \label{eq:tower}
\end{equation}
To make the expression more legible we make some abuse of notation and write $C_i \sim W$ in place of $\tilde{C} \sim W$ and write $L \sim \lambda_i$ to denote that $L$ takes values in $\mathcal L_i$ according to $\lambda_i$ given $\tilde{C} = C_i$. And thus our objective is reduced to
\[
	OPT_{LP} = \max_{\lambda_{\mathcal{C}}} \underset{C_i \sim W}{\Ex} \left[ \underset{L \sim \lambda_{i}}{\Ex} \left[ R_i( L (S_i )) |C_i \right] \right]
\]
where $R_i( L (S_i ))$ is an indicator whether assignment $L$ for scope $S_i$ satisfies the relation given by $R_i$.
Having obtained a valid LP relaxation for any CSP problem, the question is now how to use this relaxation to generate good feasible solutions to the CSP. 
A key technique to generate CSP solutions is rounding the LP solution.


\subsection{A Rounding Scheme for the Canonical LP Relaxation}
Various techniques have been developed to extract good CSP solutions from Basic LP.
A common technique to generate CSP solutions with a performance guarantee is randomized rounding.
This technique uses the information available from the LP, typically by using the optimal solution as a probability distribution, to randomly round the each variable to a value in its domain $D$.
To showcase a potential rounding technique, let us consider the problem of Max $k$-SAT. 

\subsubsection{Randomized Rounding scheme for LP relaxation of Max k-SAT}\label{sec:lpRoundingSat}
This rounding scheme was introduced by Goemans and Williamson \cite{GoeWil94}. They combine this randomized rounding scheme together with the seminal approximation algorithm for Max-SAT by Johnson \cite{Joh73} to obtain a $\frac{3}{4}$ bound.
 
Consider the full assignment $F$ as a vector of random variables. For each variable $v \in V$, let the random variable be given by:
\[
	F(v) = \begin{cases}
	1 & \text{with probability } \mu^*_v[1]\\
	0 & \text{with probability } \mu^*_v[0]
	\end{cases}
\]
where $\mu^*_v[\ell]$ is the value of variable $\mu_v[\ell]$ in an optimal solution to the LP.
Since the sum over $\ell \in D = \{0,1\}$ of  $\mu_v[\ell]$ is $1$, and since $\mu_v[\ell]$ is non-negative, the above is a valid definition for a random variable. 
Furthermore, note that $F(1), \dots, F(n)$ is a feasible solution to the Max k-SAT problem.

The expected objective value of this randomized assignment is given by taking expectation in \ref{expfirst} and we get
\[
	\mathbb{E}_F  \left[\text{Val}_{\mathcal{C}}[F] \right]= \mathbb{E}_F \mathbb{E}_{\tilde{C}}\left[R_F(\tilde{C})\right]
\]
Since we can swap the order of expectation, this is equivalent to
\[
	 \mathbb{E}_{\tilde{C}} \mathbb{E}_F \left[R_F(\tilde{C})\right]
\]
and under our new notation we replace $\tilde{C}$ with $C_i$ to get 
\[
	 \mathbb{E}_{C_i \sim W} \mathbb{E}_F \left[R_F(C_i)\right].
\]
Now note that, since $R_F(C_i) = \mathbbm{1} [\mbox{$F$ satisfies $C_i$]}$ we get $\mathbb{E}_F \left[R_F(C_i)\right] = \mathbb{P}_F [\mbox{ $F$ satisfies $C_i$}]$ and our objective function becomes
\[
\mathbb{E}_F  \left[\text{Val}_{\mathcal{C}}[F] \right] = \underset{C_i \sim W}{\mathbb{E}}\left[ \underset{F}{\mathbb{P}}[F \text{ satisfies } C_i] \right]
\]
This swap allows us to consider each $C_i \in \mathcal{C}$ separately.
Therefore, a single constraint $C_i \in \mathcal{C}$ can be considered.

Observe that since the constraints are in disjunctive form, each literal needs to evaluate to false for the constraint to be false. 
In other words, there is a unique falsifying assignment $b_{C_i}$ that makes constraint $C_i$ false. 
For example, if $C_i = x_1 \vee x_2 \vee \bar{x}_3$, then the unique falsifying assignment is given by $x_1 = 0$, $x_2 = 0$, and $x_3 = 1$. 
Considering the definition of $F(v)$ and the independence of $F(i)$ and $F(j)$ for $i \neq j$, it then follows that 
\begin{equation}
		\underset{F}{\mathbb{P}}[ F \text{ satisfies } C_i] = 1 - \prod_{v \in S_i} \mu_v[b_{C_i}(v)] \label{eq:objectiveRounding}
\end{equation}
where $b_{C_i}(v)$ is the value of variable $v \in S_i$ in the falsifying assignment. 
In particular, the probability of not satisfying the constraint in the above example is $ 1- \mu_1^*[0] \mu_2^*[0] \mu_3^*[1]$.

Having identified the objective value of the randomized solution, the question is whether any guarantees can be made on the quality of the solution compared to the LP. 
This can be done, by relating $F$ to the distribution over local assignments.
\begin{align}
	p_{C_i} &:=  \underset{L \sim \lambda_{i}}\Ex \left[ R_i( L (S_i )) |C_i \right]\nonumber\\
			&= \underset{L \sim \lambda_{i}^*}{\mathbb{P}} [L \text{ satisfies } C_i] \nonumber\\
			&=  \underset{L \sim \lambda_{i}^*}{\mathbb{P}}\left[ \bigcup_{v \in S} \left\{ L(v) \neq b_{C_i}(v) \right\} \right] \nonumber\\
			&\le \sum_{v \in  S } \underset{L \sim \lambda_{i}^*}{\mathbb{P}} \left[  \left\{ L(v) \neq b_{C_i}(v) \right\} \right] \nonumber\\
			&= \sum_{v \in S} \left( 1 - \mu_v^*[b_{C_i}(v)] \right) \label{eq:objectiveLP}
\end{align}
This last step follows from the first-order consistency constraints:
\[
	\underset{L \sim \lambda_{i} }{\mathbb{P}}[ L(v) = \ell] = \underset{L \sim \mu_v }{\mathbb{P}}[ L = \ell]
\]

It remains to relate the LP objective value to the expected objective of the rounding procedure. 
Note that the former (see \eqref{eq:objectiveLP}) is an arithmetic mean (AM) whereas the latter (see \eqref{eq:objectiveRounding}) is an geometric mean (GM). 
This suggests the use of the inequality of arithmetic and geometric means.
\begin{align*}
		\underset{F}{\mathbb{P}}[ F \text{ satisfies } C_i] &= 1 - \prod_{v \in S_i} \mu_v[b_{C_i}(v)]\\
		&= 1 - \underset{v \in S_i}{GM}(\mu_{v}^*[b_c(v)])^{|S_i|}\\
		&\ge 1 - \underset{v \in S_i}{AM}(\mu_{v}^*[b_c(v)])^{|S_i|}\\
		&= 1 - \left( 1  - \underset{v \in S_i}{AM}(1 -\mu_{v}^*[b_c(v)]) \right)^{|S_i|}\\
		&\ge 1 - \left( 1  - \frac{p_{C_i}}{|S_i|}\right)^{|S_i|}\\
\end{align*}
One can show that $$1 - \left( 1  - \frac{p_{C_i}}{|S_i|}\right)^{|S_i|} \geq r \cdot p_{C_i} ~\ \forall p_{C_i} \in [0, 1]$$ and $r \downarrow 1-1/e$ as $|S_i| \rightarrow \infty$.Therefore, the objective value of the rounding scheme
\begin{align*}
	\underset{F}{\Ex} \left[ \text{Val}_{\mathcal{C}}[F]\right] &= \underset{F}{\Ex}\left[ \underset{C_i \sim W}{\mathbb{P}}[ F \text{ satisfies } C_i] \right]\\
	&=  \underset{C_i \sim W}{\mathbb{E}}\left[ \underset{F}{\mathbb{P}}[ F \text{ satisfies } C_i] \right]\\
	&\ge \left( 1 - \frac{1}{e} \right) \underset{C_i \sim W}{\Ex} [ p_{C_i} ]\\
	&= \left( 1 - \frac{1}{e} \right) \text{OPT}_{LP}(\mathcal{C})\\
	&\ge \left( 1 - \frac{1}{e} \right) \text{OPT}(\mathcal{C})\\
\end{align*}

Hence, we conclude with the following theorem:
\begin{thm}
	Randomized rounding is a $\left( ( 1 - 1/e)\beta, \beta \right)$-approximation for Max-SAT for any $\beta$.
\end{thm}
\subsection{CSP-Specific LP Relaxations and Rounding Schemes}
Note the following specific relaxations \cite{GoeWil94,Asa97,Yan94}.
\begin{itemize}
\item this thing
\item this thing as well
\item also this thing
\end{itemize}

	
	\newpage
	% sdp relaxation
	%\documentclass[12pt]{article}
%\usepackage{fullpage,marginnote,caption}
%\usepackage{bbm,amsmath,amssymb,amsthm,verbatim}

%\input shortcuts.tex

%\begin{document}

\begin{center}
\large \bf SDP Duality
\end{center}

\bigskip

%\input sdpbegin.tex
%\input sdpmain.tex

\section{Motivation}

We recall the relaxations and try to motivate our way through the various relaxations -
\begin{discussion}{\bf Integer Programming formulation of the CSP}\\
Introduce $D$ slack variables, namely $\mu_v(\ell) \in \{ 0, 1 \}, \fa \ell \in D$ for each variable $v \in V$, call them \textit{variable indicators}. 
Next, for each $C=(R,S)\in \C$, denote by $\Lm_C$, the set of all possible assignments $L : S \rightarrow D$. 
$L$ can be viewed as restriction of a full assignment $F: V \rightarrow D$ to $S$, i.e., $F|_S$\footnote{We distinguish between partial assignments and total assignments for clarity, and hence use different symbols namely $L$ for partial, and, $F$ for total assignment respectively.}. 
Next, introduce dummy variables \textit{constraint indicators} for each constraint $C=(R,S)$, namely $\la_C(L) \in \{ 0, 1\} \fa L \in \Lm_C$. 
Impose the following constraints 
\al{
\fa v \in V, \fa \ell \in D ~:~ \mu_v(\ell) &\in \{0, 1\} \label{vval}\\
 \fa C \in \C, \fa L_C \in \Lm_C ~:~ \la_C(L) &\in \{0, 1\}  \label{cval}
}
The interpretation of $\mu_v(\ell)=1$ is that variable $v$ is assigned value $\ell$, similarly $\la_C(L)=1$ stands for picking up local assignment $L$ for the constraint $C$. 
To ensure consistency we impose the following logical constraints -
 \al{
\sum_{\ell  \in D} \mu_v(\ell) &= 1\label{vass}\\
\sum_{L \in \Lm_C} \la_C(L) &= 1 \label{cass}\\
\sum_{\substack{L \in \Lm_C \\ L(v) = \ell }} \la_C(L) &= \mu_v(\ell) \label{cons}
} where (\ref{vval}) and (\ref{vass}) say that $\fa v \in V$, exactly one $\mu_v(\ell)$ is $1$, and all others are zero, which for us simply means that each variable takes exactly one of the values $\ell$ in $D$. 
Similarly (\ref{cval}) and (\ref{cass}) imply that $\fa C \in \C$ exactly one assignment $L$ is chosen from $\Lm_C$. (\ref{cons}) stands for consistency across \textit{variable indicators} and \textit{constraint indicators}.
Thus this is an exact formulation of CSP and hence is NP Hard. 
\end{discussion}

\begin{discussion}{\bf LP Relaxation}\\
LP relaxation was motivated by relaxing the values that the \textit{variable indicators} and \textit{constraint indicators} can take. 
We relax the constraints (\ref{vval}) and (\ref{cval}), but keep (\ref{vass}), (\ref{cass})  and (\ref{cons}) constraints intact. 
In place of the binary range of values $\{0, 1\}$ we now allow the \textit{indicators} to belong to the entire interval $[0, 1]$ and call them \textit{distributions} instead (for the reasons that become apparent soon).
That is, now (\ref{vass}) and (\ref{cass}) are relaxed to 
\al{
\fa v \in V, \fa \ell \in D ~:~ \mu_v(\ell) &\in [0, 1] \label{lvas}\\
\fa C \in \C, \fa L_C \in \Lm_C ~:~ \la_C(L) &\in [0, 1] \label{lcas} 
}
(\ref{lvas}) and (\ref{lcas}) combined with (\ref{vass}) and (\ref{cass}) give us the following interpretation - \\
\textit{$\fa v \in V, \mu_v(\cdot) : D \rightarrow [0, 1]$ denotes the \textit{probability distribution of variable $v$ over its range $D$}. And, $\fa C \in \C, \la_C(\cdot) : \Lm_C \rightarrow [0,1]$ denotes the \textit{probability distribution over all possible assignments $L \in \Lm_C$ for variables in a constraint}.  (\ref{cons}) imposes conditions for consistency across \textit{distribution} $\mu_v(\cdot)$ for a variable $v$ and its marginals with respect to any constraint distribution such that the scope of that constraint contains $v$.}
\end{discussion}

We can go a step further in restricting our search space, by introducing a second order consistency condition such that we get an SDP relaxation. 
The relaxation will be with respect to IP formulation, but tightening with respect to the LP relaxation and hence our values will be now closer to the optimal value of CSP. 
The intuition for such a relaxation can be motivated by the relaxation procedure for MAX - CUT problem which we present next. 

It is further established, that this relaxation is in-fact the optimal poly-time relaxation for any general CSP, first by assuming Unique Games Conjecture \cite{Khot} and then later in fact more generally \cite{nphard}. 

%\section*{MAX-CUT Problem - Revisited!}
\begin{example}{\bf MAX - CUT}\label{maxcut}\\
MAX CUT problem can be formulated as the following integer quadratic program : 
\al{
 \max \sum_{i,j} w_{ij} \ \frac{1}{2}(1-x_ix_j) ~:~ x_i \in \{ -1, 1\}
 }
Clearly $x_i^2=1$, and $x_i=x_j\rr x_ix_j=1$ and thus contribution of that term to the sum is zero. 
Also, $x_i \neq x_j \rr x_ix_j=-1$, and thus that weight contributes to the sum. 
We can see that by associating each vertex $i$ with the variable $x_i$, we have that if an edge has both end points with same $x$-value, it is ignored in the sum, and all the edges with end points in different groups contribute weight to the sum. 
[Delorme - Poljak '90] relaxed the above program by replacing each variable $x_i$ with a unit vector $y_i$ in $\R^n$ such that its norm is 1, i.e. $y_i \in S^{n-1}$, and replacing the $x_ix_j$ by the dot product between the vectors. 
\[ \max \sum_{i,j} w_{ij} \ \frac{1}{2}(1-y_i\trans y_j) ~:~ y_i\trans y_i = 1 {\mbox {\ i.e.\ }} y_i \in S^{n-1} \]
If we define a matrix $\Sigma$ such that $$\Sigma_{ij} = y_i\trans y_j,$$ then one can easily see that for $$Y = (y_1, \cdots, y_n) \rr \Sigma = Y\trans Y \succeq 0,$$ and our program reduces to 
\[ \max_\Sigma \sum_{i,j} w_{ij} \ \frac{1}{2}(1-\Sigma_{ij}) ~:~ \Sigma \succeq 0 ~:~ \Sigma_{ii} = 1  \]
which is clearly an SDP and can be solved in polynomial time, i.e. efficiently! 
Thus we see that the SDPs did pop out for a MAX CUT relaxation.
\end{example}

\section{Basic SDP Relaxation}

The SDP relaxation is similar to the LP realization but with an important generalization. 
We will have the exactly same probability distribution $\la_C$ for each constraint and the same objective function. 
However, rather than having the the marginals for the variables we will impose conditions on second order distribution of these variables. 
More precisely,
\al{
 \Pb_{L \sim \la_C}(L(v) = \ell, L(v')= \ell') = \Sigma_{\lrb{v, \ell},\lrb{v', \ell'}}, ~:~ \Sigma \succeq 0 \label{eq0}
 }
which is same as saying: 
\al{
\Pb_{x \sim \la_C} (x_i = a, x_j = b) = \Sigma'_{\lrb{i, a},\lrb{j, b}} ~:~ \Sigma' \succeq 0. 
}
Note that the two ways are equivalent, just with different notation. 
$L$ is nothing but an assignment of variables to the values, i.e., $L : V \rightarrow D$. 
We will stick to the first notation of imposing constraints over $\la_C$'s. 

The way we have defined the constraints we can see that the SDP case does cover the optimal assignment. 
If the optimal assignment was $x_i=a_i$, we can set $ \Sigma'_{\lrb{i, a},\lrb{i, a}} = 1$ and all other elements to zero, and see that $\Si' \succeq 0$. 
Thus this is indeed a relaxation.


There is a more popular but equivalent way to define these constraints, and it helps us to understand the rounding scheme that we will discuss later on. 
Note that to any positive semi-definite matrix $\Sigma$, we can associate $L$ such that, $\Si = LL\trans$, and in turn we can label $\Si$ as the covariance matrix of some joint random variables. 
Further, if we take these joint random variables as Gaussian, then we can as well make draws from them, as the first and second moments of a Gaussian  uniquely determine its distribution. 
These observations will be crucial for rounding process, but we stated them here to motivate this alternate expression of the constraints. 

We associate $\Si$ to joint real random variables $\lrb{I_v(\ell)}_{v\in V, \ell \in D}$, by calling it their covaraince matrix. Since we impose $\Si \succeq 0$, it is indeed a valid representation. 
We will also have constraints which cause these random variables to hang together with the $\la_C$'s in a gentlemanly fashion. 
These joint random variables $I_v(\ell)$ will be called as \textit{pseduoindicator random variables}. 
We emphasize that they are jointly distributed. 
One can think of them as follows: 
\textit{``There is a box, when you press a button on one side of the box i.e., when you make a draw, out comes value for each of the $|V| \cdot |D|$ random variables."}

We are ready to properly define the SDP relaxation using this random variable representation and its covariance structure.  
We denote an SDP solution by $\Sm$. 
\begin{definition}
{\bf Basic SDP for CSP$(\C)$}  
\al{
\mbox{SDPOpt}(\C) = \max_{\substack{\la_C, \Si, A}} \mbox{SDPVal}_\C(\Sm) :=\sum_{C=(R,S) \in \mathcal{C}} w_C \Pb_{L \sim \la_C} \lrbb{L(S) \mbox{\ satisfies \ } R} \text{\footnotemark} 
}
\footnotetext{A more general objective function can be defined with respect to payoff  functions $\phi_C$ associated with each constraint: $$\max_{\la_C, C\in \mathcal{C}} \sum_{C=(R,S) \in \mathcal{C}} w_C \Ex_{L \sim \la_C} \phi_C(L) = \sum_{C \in \mathcal{C}} \sum_{L \in \mathcal{L}_C} \la_C(L) \lrb{w_C\phi_C(L)}  $$}

subject to $\fa C = (R, S) \in \mathcal{C},\fa v, v' \in S, \fa \ell, \ell' \in D $
\al{
\Pb_{L \sim \la_C} (L(v) = \ell \wedge L(v') = \ell') & \os{\lrb{\ref{eq0}}}= \Sigma_{\lrb{v, \ell},\lrb{v', \ell'}} = \Ex \lrbb{I_v(\ell) \cdot I_v(\ell')} \label{csm}
}
And we also impose the conditions for valid probability distribution : 
\al{
\Pb_{L \sim \la_C} (L(v) = \ell \wedge L(v) = \ell') &= 0, \fa \ell \neq \ell' \label{eq1}\\
\sum_{\ell \in D}\Pb_{L \sim \la_C} (L(v) = \ell) &= 1 \label{eq2}
}
We also define the following constraint: (though this condition is deemed redundant by the Remark \ref{remark01}, but explicitly assuming it here makes our life easy): 
\al{
\Pb_{L \sim \la_C} (L(v) = \ell ) = \Ex[I_v(\ell)] = A_{v, \ell} \label{cfm}
}
\end{definition}

\section{How is this an SDP?}

Note that for a given $C =(R, S) \in \mathcal{C}$, $\la_C$ is a probability distribution over all possible assignments $\mathcal{L}_C = \{L: S \rightarrow D\}$. 
If we define 
\al{
\phi_C(L) = \mathbb{I}[L \mbox{ satisfies } R], \fa L \in \mathcal{L}_C,
}
then 
\al{
\Pb_{L \sim \la_C} \lrbb{L(S) \mbox{\ satisfies \ } R} = \Ex_{L\sim \la_C} \phi_C(L) = \sum_{L \in \mathcal{L}_C} \la_C(L) \phi_C(L)
}
This makes the objective explicitly linear in the variables $\la_C$:
\al{
\sum_{C \in \mathcal{C}} w_C \Pb_{L \sim \la_C} \lrbb{L(S) \mbox{\ satisfies \ } R} &= \sum_{C \in \mathcal{C}} w_C \lrbb{\sum_{L \in \mathcal{L}_C} \la_C(L) \phi_C(L)}\\
&=  \sum_{C \in \mathcal{C}} \sum_{L \in \mathcal{L}_C} \la_C(L) \lrb{w_C\phi_C(L)} 
}
Next we see that 
\al{
\Sigma_{\lrb{v, \ell},\lrb{v', \ell'}} &=\Pb_{L \sim \la_C} (L(v) = \ell \wedge L(v) = \ell') \\
&= \sum_{\substack {L\in \mathcal{L}_C\\ L(v) = \ell, L(v')=\ell'} }\la_C(L)
}
is a linear constraint in the variables. 
\al{
&\sum_{\ell \in D}\Pb_{L \sim \la_C} (L(v) = \ell) = 1 \\
\lrr &\sum_{\ell \in D}  \sum_{\substack {L\in \mathcal{L}_C\\ L(v) = \ell }}\la_C(L) = 1
}
is also linear.
It is trivial to see that the last constraint is also linear (though this is redundant for analysis):
\al{
\Pb_{L \sim \la_C} (L(v) = \ell ) = \Ex[I_v(\ell)] = A_{v, \ell} \\
\lrr   \sum_{\substack {L\in \mathcal{L}_C\\ L(v) = \ell }}\la_C(L) = A_{v, \ell} 
}
Last but not the least, the way we have defined $\Si$, we have the equivalent constraint:
\al{
\Si \succeq 0
}
Thus, with a linear objective and some linear constraints, besides an SDP constraint we do have an SDP relaxation.


\section{Some Analysis}
\begin{remark}\label{remark01}
If we drop the ``consistent first moment" condition (\ref{cfm}) from the Basic SDP, we get
an equivalent formulation; i.e., any solution $\mathcal{S}$ to the Basic SDP which doesn't satisfy  (\ref{cfm}) can be transformed into a solution $\mathcal{S}'$ which does satisfy  (\ref{cfm}) and has $\mbox{SDPVal}_\C(\Sm')
= \mbox{SDPVal}_\C (\Sm)$.
\end{remark}

\begin{remark}
The consistent first and second moment constraints imply 
\al{
\Ex \lrbb{I_v(\ell)} = \Ex \lrbb{I_v(\ell)^2}
}
\end{remark}
\begin{proof}
\al{
\Ex \lrbb{I_v(\ell)^2)} &= \Pb_{L \sim \la_C} (L(v) = \ell \wedge L(v) = \ell) \\
&= \Pb_{L \sim \la_C} (L(v) = \ell ) \\
&= \Ex[I_v(\ell)] 
}
\end{proof}
\begin{remark}
\al{
 \sum_{\ell \in D} I_v(\ell) = 1 \quad \mbox{a.s.}, ~ \fa v \in V
 }
\end{remark}
\begin{proof}
Define $J_v = \sum_{\ell \in D} I_v(\ell)$, then 
\al{
\Ex[J_v] = \sum_{\ell \in D} \Ex [I_v(\ell)] =  \sum_{\ell \in D} \Pb_{L \sim \la_C} (L(v) = \ell ) = 1
}
\al{
\Ex[J_v^2] &= \Ex \lrbb{\lrb{ \sum_{\ell \in D} \Ex [I_v(\ell)]}\lrb{ \sum_{\ell' \in D} \Ex [I_v(\ell')]}} \\
&= \sum_{\ell, \ell' \in D} \Pb_{L \sim \la_C} (L(v) = \ell \wedge L(v) = \ell') \\
&\os{\lrb{\ref{eq1}}}= \sum_{\ell \in D}\Pb_{L \sim \la_C} (L(v) = \ell) \\
&\os{\lrb{\ref{eq2}}}=1\\
&=\Ex[J]^2
}
Thus we have 
\al{
\mbox{Var}(J_v) = \Ex[J_v^2] - \Ex[J_v]^2 = 1-1 = 0 \rr J_v = J  \equiv 1 \ \mbox{a.s.}
}
\end{proof}

\section{Intuition for Rounding}
Next question that comes to our mind is - 
\textit{``How do we get a near optimal assignment from the solution of the relaxed problem"}. 
Note that the solution of the relaxed problem gives us a PSD matrix and probability distributions over constraints. 
Next challenge is to make a ``consistent draw" from this distribution and analyzing the expected value of CSP resulting from such draws, 
however we have already understood the intuition behind various terms and hence it is merely a computational task now.  

We concretely summarize various angles, that we discussed briefly before, to look at canonical SDP relaxations for a CSP :
\iit{

\item Pseudo-indicator random variables which satisfy the first and second moment consistency constraints. 
This perspective is arguably the best for understanding the SDP. 

\item Pseudo-indicator random variables which satisfy the second moment constraints. 
This perspective is arguably the best when constructing SDP solutions by hand.

\item Vectors $\{ y_{v, \ell} \}$ satisfying the first and second moment consistency constraints.  
This perspective is the one that's actually used computationally, on a computer. 

\item Jointly Gaussian Pseudo-indicator random variables which satisfy the consistent first and second moment constraints. 
This perspective is suited for developing``SDP Rounding algortihms". 
}
Let's discuss a classical rounding scheme for MAX-CUT and then introduce a more general rounding scheme - 
\begin{example}{\bf Randomized Rounding [GW Rounding]}\\
By now, we know how to relax MAX-CUT to an SDP, and assuming efficient poly-time algorithm for SDP solver, we have a solution for the SDP formulation. Now, we want to somehow convert that back to \textit{good solution} for Max-cut. 
We will borrow notation and variables from Example \ref{maxcut}.

We assume that given $\Si$ we can find\footnote{We discuss this in appendix.} $Y : \Si = Y\trans Y$. 
We want to cut the vectors $\{y_i\}_{i \in V}$ with a random hyper plane through origin such that all vectors on one side of he hyperplane correspond to variables in one partition, and the rest are classified into other partition, thereby giving us a cut. 
We do this by choosing a vector $\hat{n} \in \R^n$ (where $\hat{n}$ denotes the normal to the hyperplane) from any rotationally symmetric distribution. Then set, $x_i = \sgn{\hat{n} \cdot y_i} \in \{-1, 1 \}$.

{\bf Analysis of the Rounding} 
For a given graph $G = (V, E)$, with $V =\{1, \ldots, n\}$, and $E$ the set of edges, fix $(i, j) \in E$. Then the probability that the edge $(i, j)$ is cut by the hyperplane is same as the probability that hyperplane splits $y_i$, and $y_j$. 
Now consider just simply the $2D$ plane containing $y_i, y_j$. Since the hyperplane was chosen from a rotationally symmetric distribution, the probability that it cuts these two vectors is same as that a random diameter in the circle containing $y_i, y_j$, lies in between the angle $\theta$ of these two vectors.

Thus, 
\al{
\Pb[(i, j) \mbox{ gets cut} ] &= \frac{\theta}{\pi} \\
&=\frac{{\rm cos\inv}\lrb{y_i \cdot y_j}}{\pi}\\
&=\frac{{\rm cos\inv}\lrb{\Si_{ij}}}{\pi}\\
\Ex[\mbox{cut val}] &= \sum_{(i, j) \in E}w_{ij} \frac{{\rm cos\inv}\lrb{\Si_{ij}}}{\pi}
}
Now recall that 
\al{
\mbox{SDPOpt} = \sum_{(i, j) \in E} w_{ij} \lrb{\frac{1}{2}-\frac{1}{2}\Si_{ij}} \geq \mbox{Opt}.
}
So, if we find $\alpha$ such that 
\al{
\frac{{\rm cos\inv}\lrb{\Si_{ij}}}{\pi} = \alpha\ \lrb{\frac{1}{2}-\frac{1}{2}\Si_{ij}} \fa \Si_{ij} \in [-1, 1]
}
then we can conclude 
\al{
\Ex[\mbox{cut val}]\geq \alpha \mbox{SDPOpt} \geq \alpha \mbox{Opt}
}
Plotting the above, ...........
we see that $\alpha = 0.87856...$ works.
\begin{remark}
$\Ex[\mbox{Goemans Williamson Cut}]  \geq 0.87856 \mbox{\ SDPOpt} \geq 0.87856 \mbox{\ Opt }$
\end{remark} 
\end{example}
 
\begin{definition}{\bf Gaussian Rounding}\\
We can $\lrb{1-O\lrb{\sqrt{\epsilon \log q}}, 1-\epsilon}$-approximate $UG_q$ by SDP Rounding \cite{cmm06}. We can do the following :
\iit{
\item Solve the canonical SDP relaxation to get a collection of jointly Gaussian pseudo-indicators $\{G_v(\ell) \}_{v \in V, \ell \in D}$
\item Draw once from them to obtain numbers $(g_v[0], \ldots, g_v[q-1])_{v \in V}$.
\item Output the assignment $F(v) = \mbox{argmax} \{g_v(\ell)\}$.
}
Note that this is a randomized algorithm.
\end{definition}
\begin{remark}
If $\mbox{SDPOpt}(\C) \geq 1- \epsilon$, then 
\al{
\Ex_{F} \lrbb{\mbox{Val}_C(F)} \geq 1 - O \lrb{\sqrt{\epsilon \log q}}
}
\end{remark}
%\end{document}
	\newpage
	%rounding schemes
	\section{Rounding Schemes for SDP Relaxations}
\subsection{Intuition for Rounding}
Next question that comes to our mind is - 
\textit{``How do we get a near optimal assignment from the solution of the relaxed problem"}. 
Note that the solution of the relaxed problem gives us a PSD matrix and probability distributions over constraints. 
Next challenge is to make a ``consistent draw" from this distribution and analyzing the expected value of CSP resulting from such draws, 
however we have already understood the intuition behind various terms and hence it is merely a computational task now.  

We concretely summarize various angles, that we discussed briefly before, to look at canonical SDP relaxations for a CSP :
\iit{

\item Pseudo-indicator random variables which satisfy the first and second moment consistency constraints. 
This perspective is arguably the best for understanding the SDP. 

\item Pseudo-indicator random variables which satisfy the second moment constraints. 
This perspective is arguably the best when constructing SDP solutions by hand.

\item Vectors $\{ y_{v, \ell} \}$ satisfying the first and second moment consistency constraints.  
This perspective is the one that's actually used computationally, on a computer. 

\item Jointly Gaussian Pseudo-indicator random variables which satisfy the consistent first and second moment constraints. 
This perspective is suited for developing``SDP Rounding algorithms". 
}
Let's begin with a classical rounding scheme for MAX-CUT and and follow with a more general SDP rounding scheme for Unique Games. In the next section, we comment on some of the claimed optimal rounding schemes for SDP relaxations of any CSP.

\subsection{Goemans Williamson Algorithm}
By now, we know how to relax MAX-CUT to an SDP, and assuming efficient poly-time algorithm for SDP solver, we have a solution for the SDP formulation. Now, we want to somehow convert that back to \textit{good solution} for Max-cut. 

We will borrow notation and variables from Example \ref{maxcut}.
We assume that given $\Si$ we can find\footnote{We discuss this in appendix.} $Y : \Si = Y\trans Y$. 
Next we discuss the rounding discussed in \cite{gwFirstMaxCutSDP}. 

We want to cut the vectors $\{y_i\}_{i \in V}$ with a random hyper plane through origin such that all vectors on one side of he hyperplane correspond to variables in one partition, and the rest are classified into other partition, thereby giving us a cut. 

{\bf Algorithm}
\begin{enumerate}
\item Draw a vector $\hat{n} \in \R^n$ (where $\hat{n}$ denotes the normal to the hyperplane) from any rotationally symmetric distribution. 
\item Set $x_i = \sgn{\hat{n} \cdot y_i} \in \{-1, 1 \}$.
\end{enumerate}

{\bf Analysis of the Rounding} \\
For a given graph $G = (V, E)$, with $V =\{1, \ldots, n\}$, and $E$ the set of edges, fix $(i, j) \in E$. Then the probability that the edge $(i, j)$ is cut by the hyperplane is same as the probability that hyperplane splits $y_i$, and $y_j$. 
Now consider just simply the $2D$ plane containing $y_i, y_j$. Since the hyperplane was chosen from a rotationally symmetric distribution, the probability that it cuts these two vectors is same as that a random diameter in the circle containing $y_i, y_j$, lies in between the angle $\theta$ of these two vectors. Thus, 
\al{
\Pb[(i, j) \mbox{ gets cut} ] &= \frac{\theta}{\pi} \\
&=\frac{{\rm cos\inv}\lrb{y_i \cdot y_j}}{\pi}\\
&=\frac{{\rm cos\inv}\lrb{\Si_{ij}}}{\pi}\\
\Ex[\mbox{Weight of edges cut}] &= \sum_{(i, j) \in E}w_{ij} \frac{{\rm cos\inv}\lrb{\Si_{ij}}}{\pi}
}
Now recall that 
\al{
\mbox{SDPOpt} = \sum_{(i, j) \in E} w_{ij} \lrb{\frac{1}{2}-\frac{1}{2}\Si_{ij}} \geq \mbox{Opt}.
}
So, if we find $\alpha$ such that 
\al{
\frac{{\rm cos\inv}\lrb{\Si_{ij}}}{\pi} = \alpha\ \lrb{\frac{1}{2}-\frac{1}{2}\Si_{ij}} \fa \Si_{ij} \in [-1, 1]
}
then we can conclude 
\al{
\Ex[\mbox{cut val}]\geq \alpha \mbox{SDPOpt} \geq \alpha \mbox{Opt}
}
Solving it numerically we get that $\alpha = 0.87856$ works.
\begin{remark}
$\Ex[\mbox{Cut}]  \geq 0.87856 \mbox{\ SDPOpt} \geq 0.87856 \mbox{\ Opt }$
\end{remark} 
 
 \subsection{Gaussian Rounding}

We can see Unique Games as Constraint Satisfaction Problems that are a generalization of Max-Cut to a large domain size. Let us define an equivalent definition of Unique Games to make this relation precise:
\begin{definition}{\bf Unique Game}\\
A unique game consists of a constraint graph $G = (V, E)$, an assignment function $F: V \rightarrow D$ and a set of permutations $\pi_{vv'}$ on $D = \{0, \ldots, q-1 \}$ (for all edges $(v, v')$). Each permutations $\pi_{vv'}$ defines the constraint $\pi_{vv'}(F(v'))=F(v)$. The goal is to find an assignment $F$ soas to maximize the number of satisfied constraints.
\end{definition}
Thus, max-cut is a unique game with $q=2$ and vice-versa, unique games is a generalization of max-cut to larger domain size. 

Next we discuss the rounding scheme from \cite{cmm06}, which is a generalization of the Goemans Williamson Algorithm for the Max-Cut to a Rounding Algorithm for Unique Games. 

Given a Unique Game $G = (V,E)$ with edge weights $W = \{w_{vv'} | (v, v') \in E \}$, we formulate it as a CSP with constraints $\C$ and weights $W$. Then, we solve the SDP relaxation for this CSP instance and decompose the solution $\Sigma = U\trans U$ with $U \in \R^{N \times N}$ with $N = |V| \cdot |D|$, using Cholesky decomposition (which is polytime). 

Denote by $[x]_r$ the function that rounds $x$ up or down depending on whether the fractional part of $x$ is greater or less than $r$. 
Note that if $r$ is uniformly distributed in the interval $[0,1]$, then the expected value of $[x]_r$ is $x$. 

{\bf Algorithm}
\begin{enumerate}
\item Pick a number $r$ in the interval $[0, 1]$ uniformly at random.
\item Pick random independent Gaussian vectors $g_1, \ldots, g_{2q}$ with independent components distributed as $\mathcal{N}(0, 1)$.
\item For each vertex $v$:
\begin{enumerate}
\item Find normalized vectors $\tilde{u}_{\ell} = {u_{\ell}}/{||u_\ell||_2^2}$
\item Set $s_{u_\ell} = [2q \cdot ||u_\ell||_2^2]_r$, for $\ell=0, \ldots, q-1$.
\item For each $\ell$, project $s_{u_\ell}$ vectors $g_1, \ldots, g_{s_{u_\ell}}$ to $\tilde{u}_\ell$:
\al{
\xi_{u_\ell, s} = g_s\trans \tilde{u}_\ell, 1 \leq s \leq s_{u_\ell}.
}
Hence for each variable $u$, there are $s_{u_0}+\ldots+s_{u_{q-1}}$ many $\xi$'s, call this set $\Xi_u$.
\item For each $u$, find the maximum magnitude member in $\Xi_u$ and let it be $\xi_{u_{\ell^*}, s^*}$. Assign
\al{
F(u) = \ell^*
}
\end{enumerate}
\end{enumerate}

\begin{theorem}
If the optimal solution of the unique game satisfies $1-\epsilon$ fraction of constraints, then the above algorithm in expectation satisfied $1-O({\sqrt{\epsilon \log q }})$ fraction of constraints.
\end{theorem}
Proof of the above theorem is quite involved and we refer the reader to \cite{cmm06} for the analysis.

	\newpage
	%\documentclass[letterpaper, 12pt]{article}
%
%\usepackage[margin=2.5cm]{geometry}
%\usepackage{amsmath,amsthm,amssymb}
%\usepackage[]{mathtools}
%\usepackage[]{bbm}
%
%\numberwithin{equation}{section}
%
%% --to donotes
%\usepackage{xargs}                      % Use more than one optional parameter in a new commands
%\usepackage[dvipsnames]{xcolor}
%\usepackage[colorinlistoftodos,prependcaption,textsize=tiny]{todonotes}
%\newcommandx{\unsure}[2][1=]{\todo[linecolor=red,backgroundcolor=red!25,bordercolor=red,#1]{#2}}
%\newcommandx{\change}[2][1=]{\todo[linecolor=blue,backgroundcolor=blue!25,bordercolor=blue,#1]{#2}}
%\newcommandx{\info}[2][1=]{\todo[linecolor=OliveGreen,backgroundcolor=OliveGreen!25,bordercolor=OliveGreen,#1]{#2}}
%\newcommandx{\maybeinclude}[2][1=]{\todo[linecolor=Orange,backgroundcolor=Orange!25,bordercolor=Orange,#1]{#2}}
%\newcommandx{\improvement}[2][1=]{\todo[linecolor=Plum,backgroundcolor=Plum!25,bordercolor=Plum,#1]{#2}}
%
%\usepackage[]{thmtools}
%\usepackage[dvipsnames]{xcolor}
%\declaretheoremstyle[
%	bodyfont=\normalfont, 
%	spaceabove=0.5cm]{defFormat}
%\declaretheoremstyle[
%	postheadspace=1cm,
%	bodyfont=\normalfont, 
%	spaceabove=0.5cm]{inLineDefFormat}
%\declaretheoremstyle[
%	spaceabove=0.5cm, 
%	spacebelow=0.5cm,
%	postheadspace=1cm]{namedTheorem}
%\declaretheorem[
%	numberwithin=section, 
%	style=defFormat, 
%	shaded]{definition}
%\declaretheorem[
%	numbered=no, 
%	style=defFormat, 
%	shaded]{algorithm}
%\declaretheorem[
%	style=inLineDefFormat, 
%	sibling=definition,
%	shaded,
%	name=Definition]{ILdefinition}
%\declaretheorem[
%	numbered=no, 
%	style=namedTheorem, 
%	shaded, 
%	name=The Unique Games Conjecture]{ugc}
%\newtheorem{thm}{Theorem}
%\declaretheorem[numberwithin=section, style=defFormat, shaded]{Definition}
%\begin{document}
\section{``Efficient" One-Size-Fits-All Optimal Algorithms}
In 2008, Prasad Raghavendra published a paper entitled \textit{Optimal Algorithms and Inapproximability Results for Every CSP?}. We highlight the important characteristics below:

\begin{itemize}
\item Raghavendra proposed an SDP-based rounding scheme for use on \textit{any} CSP.
\item The performance guarantees cannot be stated in the usual ``$\alpha$-approximation," or even ``$(\alpha,\beta)$-approximation" sense.  These guarantees are called \textit{non-explicit}.
\item If the Unique Games Conjecture were true, then the proposed algorithm would be \textit{optimal} in that no polynomial time algorithm could provide stronger performance guarantees.
\item The proof is unusual in that if the UGC does not hold, then performance guarantees \textit{disappear} for CSP's with arity greater than 2.
\item The algorithm does have some performance guarantees for 2-CSP's irrespective of the truth of the UGC.
\end{itemize}

The second of the points above seems truly profound. It suggests that even purpose-built approximation algorithms could have asymptotically inferior performance to Raghavendra's generic algorithm. Indeed, many using optimization in practice would want to know: is Raghavendra's generic algorithm more computationally expensive than some purpose-built algorithms? Does Raghavendra's algorithm outperform purpose-built algorithms in practice? 

To add to these questions, Raghavendra and Steurer proposed yet another rounding scheme in their 2009 paper \textit{How to Round Any CSP}. Again, we highlight the important characteristics:

\begin{itemize}
\item They propose a generic SDP-based rounding scheme for use on any CSP, with accompanying performance guarantees that are independent of the truth of the UGC.
\item As before, the performance guarantees of the proposed algorithm are non-explicit
\item Their algorithm is ``polynomial time" and yet ``runs in time $\exp{(\exp{(\text{poly}(kq/\epsilon)}))}$".
\end{itemize}

The first point is at first very exciting, but the third is somewhat baffling. Where do these double-exponentials come from, and how is it that this algorithm could possibly be ``polynomial time?"

The remainder of this section is devoted to clearly communicating how these two algorithms would work if they where in fact implemented. Along the way, we will see the extent to which implementation is even possible.
\subsection{Constructing $\text{SDP}_{\text{gen}}$ (used in both papers)}
In \textit{Optimal Algorithms ... for Every CSP?}, the following SDP was called SDP(III). In subsequent work (\textit{How to Round Any CSP}) the following SDP was called $\text{SDP}_{\text{gen}}$. We use the newer term here to prevent the misconception that these SDP's are different.

\subsection{Optimal Algorithms ... for Every CSP? (2008)}
The reader is advised that this section is organized in a very deliberate way. We present the algorithm at a high level, and then we address each bullet point in detail. We intentionally explain the steps of the algorithm in \textit{increasing order of technical difficulty}, rather than in the order that they would be executed during runtime. Once all this is presented, we have a brief discussion of time complexity and performance guarantees.

This established, we move on to the algorithm itself. In the original paper, the following algorithm is only known as ``Round." We give it a proper\footnote{although not necessarily \textit{good}} name out of necessity.

\begin{algorithm} \textbf{UGDFS} : \textbf{U}nique \textbf{G}ames \textbf{D}ependant \textbf{F}unction \textbf{S}earch \\

\textit{Input: } An instance $\mathcal{C}$ of $\text{CSP}(\Gamma)$, as well as parameters $\kappa$ and $K$.

\textit{Output: } An assignment $F$ of variables $V \in \mathcal{C}$.
\begin{itemize}
\item Build an SDP $\text{SDP}_{\text{gen}}(\mathcal{C})$.
\item Solve $X \leftarrow \text{SDP}_{\text{gen}}(\mathcal{C}).\text{solve}$.
\item Smooth the SDP solution $X \leftarrow \text{Smooth}(X)$
\item Discretize a certain space of functions ``$\mathcal{S}$" to form $\mathcal{S}_{\kappa}$.
\item For every function $\mathcal{F} \in \mathcal{S}_{\kappa}$, run a subroutine called ``$\text{Round}_{\mathcal{F}}(X)$" to get an assignment $F$ of the variables in $\mathcal{C}$. 
\item Return the best assignment generated over all of these $\mathcal{F} \in \mathcal{S}_{\kappa}$.
\end{itemize}
\end{algorithm}

\subsubsection{What is $\Omega$? How Do We Discretize It?}

\subsubsection{The $\text{Round}_{\mathcal{F}}$ Subroutine}

\subsubsection{Smoothing an SDP Solution}

\subsubsection{Time Complexity of UGDFS}

\subsubsection{Performance Guarantees of UGDFS}

\subsection{How to Round Any CSP (2009)}
As before, we present the algorithm at a high level and then address each bullet point in detail. Unlike the previous section, we explain the steps in the order that they would be executed by the algorithm. The reader is welcome to read these explanatory sub-sections in any order. 

Discussion of runtime and performance guarantees follows the step explanations.


\begin{algorithm} \textbf{The Variable Folding Method} \\

\textit{Input: } An instance $\mathcal{C}$ of $\text{CSP}(\Gamma)$, as well as parameters $\epsilon$, THINGS.

\textit{Output: } An assignment $F$ of variables $V \in \mathcal{C}$.
\begin{itemize}
\item Build $\text{SDP}_{\text{gen}}(\mathcal{C})$.
\item Solve $X \leftarrow \text{SDPgen}(\mathcal{C}).\text{solve}$.
\item SMOOTHING???
\item Identify a set of ``bad" constraints $B_{\epsilon} = \text{ConstraintAssesor}(\mathcal{C},X)$.
\item Define a new CSP, $\mathcal{C}'$ with constraint set $\{C_i \in \mathcal{C} : C_i \not \in B_{\epsilon}\}$.
\item Define yet another CSP, ``$\mathcal{C}'/\phi$" $ \leftarrow \text{Folding}(\mathcal{C}')$.
\item Find an \textit{\textbf{optimal}} variable assignment $F^*$ for $\mathcal{C}'/\phi$ with any \textit{\textbf{exact}} algorithm.
\item Construct $F^{**} = \text{Unfolding}(F^*, \mathcal{C}'/\phi)$ - a variable assignment for $\mathcal{C}'$
\item Use $F^{**}$ as an assignment of variables for $\mathcal{C}$.
\end{itemize}
\end{algorithm}

\subsubsection{Smoothing an SDP Solution (revisited)}

\subsubsection{Identifying ``Bad" Constraints}

\subsubsection{Constructing $\mathcal{C}'/\phi$ by ``Folding"}

\subsubsection{Computing an Optimal Variable Assignment for $\mathcal{C}'/\phi$}

\subsubsection{Unfolding the Variable Assignment $F^*$}

\subsubsection{Time Complexity of The Variable Folding Method}

\subsubsection{Performance Guarantees of The Variable Folding Method}

%\end{document}
	\newpage
	\section{Numerical Experiments}
In this section, the LP and SDP relaxations as well as rounding schemes introduced in sections \label{sec:lpRelax} and \ref{sec:sdpRelax} are put to the test for key CSP problems, such as Max-Cut and Max k-SAT. This section provides empirical validation and identifies their usefulness in practical applications. In particular, this section contains numerical studies for the LP relaxation and the Max k-SAT rounding scheme and the SDP relaxation in combination with the Goeman-Williamson rounding scheme for Max Cut. 

\subsection{LP Relaxation and Rounding scheme for Max k-SAT}
The LP relaxation as introduced in section \ref{sec:lpRelax} can be applied to any CSP. However, it does not provide a feasible solution to the CSP ( unless the optimal solution happens to be feasible for CSP) since the LP is a relaxation of a CSP. Therefore, its application is dependent on a technique to generate feasible solutions from the LP solution. Such a technique was introduced for K-SAT in section \ref{sec:lpRoundingSat}. In this section, results are presented of applying the LP relaxation and the rounding scheme to Max 3-SAT on randomly generated 3-SAT instances.

The 3-SAT instances where generated as follows:
\begin{enumerate}
	\item A solution $x^*$ on $k$ variables is generated. Each variable is independently set to $1$ or $0$, where $1$ is chosen with probability $p_T$. $p_T$ is the probability of setting a variable to true. This is probability is drawn from a uniform distribution on $[0,1]$. 
	\item $m$ constraints are added. These constraints are constructed via the following procedure:
	\begin{enumerate}
		\item The number of negated variables in the scope is uniformly drawn from $[0,1,2,3]$.
		\item 
	\end{enumerate}
\end{enumerate}

To be finished.

RESULTS

\begin{table}
	\footnotesize
	\centering
	\begin{tabularx}{\textwidth}{>{\centering}p{1.7cm}>{\centering}p{1.7cm}ssCCmC}
		\toprule
		Number of variables & Number of constraints & Mean $z^*$ (s.d.) & Mean $z^*_{LP}$ (s.d.) & Mean $z^*_{round}$ (s.d.) & Mean $t_{gen}$ (s.d.) & Mean $t_{solve}$ (s.d.) & Mean $t_{round}$ (s.d.) \\ \midrule
		10                  & 20                    &     1 (0)     &           1 (0)            &                 0.885 ( 0.08)                  &         0.02 ( 0.03)          &      1.48 (0.07)                      &  0.00 (0.00)\\
		10                  & 50                    &    1   (0)    &           1 (0)            &                  0.887 (0.05)                  &         0.05 ( 0.01)          &       3.5 (0.3 )                     &  0.00 (0.00)\\
		20                  & 40                    &     1 (0)     &           1 (0)            &                  0.880 (0.05)                  &         0.039 (0.004)         &       4.8 (0.1)                     & 0.00 (0.00) \\
		20                  & 100                   &     1 (0)     &           1 (0)            &                  0.891 (0.04)                  &          0.13 (0.02)          &       11.9 (0.9)                     & 0.00 (0.00) \\
		50                  & 50                    &    1  (0)     &           1 (0)            &                  0.892 (0.04)                  &          0.06 (0.01)          &       14.4 (0.7)                     &  0.00 (0.00)\\
		50                  & 100                   &    1   (0)    &           1 (0)            &                  0.891 (0.03)                  &          0.14 (0.01)          &       28.4 (0.4)                &  0.00 (0.00)\\
		50                  & 200                   &    1   (0)    &           1 (0)            &                  0.882 (0.02)                  &          0.34 (0.02)          &       51 (1)                     & 0.00 (0.00) \\
		100                 & 100                   &    1   (0)    &           1 (0)            &                  0.899 (0.04)                  &          0.13 (0.02)          &        50 (1)                     &  0.00 (0.00)\\
		100                 & 200                   &    1  (0)     &           1 (0)            &                  0.883 (0.02)                  &           0.4 (0.2)           &        122 (22)                    &  0.00 (0.00)\\ \bottomrule
	\end{tabularx}                                                                                                                                  
\end{table}

\subsection{Experiments with Max-Cut}
In this section, we generate several max-cut problem instances and then compare the results using LP Relaxation and SDP Relaxation. And we also present empirical results for the schemes namely LP Rounding and GW Rounding. 

The Max-Cut instances were generated as follows. We provide set of vertices $(V)$, and the number of edges as the input $(E)$. Next, 
\begin{enumerate}
	\item We randomly partition the variable set into two disjoint subsets. For $|V|$ variables, we do this by selecting uniformly at random $[|V|/2]$ variables independently from $V$ and declaring them as one set and its complement as the other set, call them $V_1$ and $V_2$. 
	\item We generate a weight vector of size $E$, with all entries drawn i.i.d. from a Uniform distribution on $[0, 1]$. We normalize the weights so that they sum to $1$. Call this $\{ w_i : i = 1, \ldots, E\}$.
	\item For $i=1, \ldots, E$ 
	\begin{enumerate}
		\item  Select uniformly at random one node each from the sets $V_1$ and $V_2$ independently. 
		\item Declare an edge between these two nodes and set the edge weight to $w_{i}$.
	\end{enumerate}
\end{enumerate}
In this fashion, we are able to generate a random instance of a bipartite graph with random edge weights that sum to $1$. The way we generate theall instances have an optimal assignment and  hence in either relaxation schemes, the optimal value will be $1$. Hence, we only compare the results for the effectiveness of the two rounding schemes, by listing the value that we obtain by randomly rounding the solution that we obtain from the relaxations. 



\begin{table}
	\footnotesize
	\centering
	\begin{tabularx}{\textwidth}{>{\centering}p{1.7cm}>{\centering}p{1.7cm}ssCCmC}
		\toprule
		
		Number of variables & Number of constraints & Mean $z^*$ (s.d.) & Mean $z^*_{LP}$ (s.d.) & Mean $z^*_{round}$ (s.d.) & Mean $t_{gen}$ (s.d.) & Mean $t_{solve}$ (s.d.) & Mean $t_{round}$ (s.d.) \\ \midrule
    5 					& 4    						& 1(0) & 1(0) 		& 0.5(0.3)		& 0.47(0.03)	& 0.0022(0.0004)   	& 0.00(0.00)\\
    10					& 50  						& 1(0) & 1(0) 		& 0.5(0.1)		& 0.89(0.02)	& 0.016(0.003)	  	& 0.00(0.00)\\
    20					& 40  						& 1(0) & 1(0) 		& 0.50(0.08)	& 0.8(0.2)		& 0.012(0.003)	  	& 0.00(0.00)\\
    20					& 100 						& 1(0) & 1(0) 		& 0.50(0.07)	& 1.3(0.1)		& 0.026(0.004)	  	& 0.00(0.00)\\
    50					& 50  						& 1(0) & 1(0) 		& 0.51(0.07)	& 0.88(0.1)		& 0.015(0.002)	  	& 0.00(0.00)\\
    50					& 100 						& 1(0) & 1(0) 		& 0.51(0.05)	& 1.30(0.04)	& 0.025(0.003)	  	& 0.00(0.00)\\
    50					& 200 						& 1(0) & 1(0) 		& 0.50(0.04)	& 2.3(0.3)		& 0.05(0.02)	  	& 0.00(0.00)\\
    100 				& 100	 					& 1(0) & 1(0) 		& 0.51(0.06)	& 1.38(0.06)	& 0.025(0.003)	  	& 0.00(0.00)\\
    100 				& 200	 					& 1(0) & 1(0) 		& 0.50(0.04)	& 2.4(0.2)		& 0.0461(0.003)	  	& 0.00(0.00)\\
 \bottomrule
	\end{tabularx}                                                                                                                                  
\end{table}




\begin{table}
	\footnotesize
	\centering
	\begin{tabularx}{\textwidth}{>{\centering}p{1.7cm}>{\centering}p{1.7cm}ssCCmC}
		\toprule
		

		Number of variables & Number of constraints & Mean $z^*$ (s.d.) & Mean $z^*_{GW}$ (s.d.) & Mean $z^*_{round}$ (s.d.) & Mean $t_{gen}$ (s.d.) & Mean $t_{solve}$ (s.d.) & Mean $t_{round}$ (s.d.) \\ \midrule
    5 					& 4    						& 1(0) & 1(0) 		& 1(0)				& 0.46(0.03)			& 0.0022(0.0004) & 0.00(0.00) \\
    10					& 50  						& 1(0) & 1(0) 		& 1(0)				& 0.46(0.04)			& 0.016(0.003)	 & 0.00(0.00)\\
    20					& 40  						& 1(0) & 1(0) 		& 1.00(0.03)		& 0.40(0.12)			& 0.012(0.003)	 & 0.00(0.00)\\
    20					& 100 						& 1(0) & 1(0) 		& 1(0)				& 0.34(0.04)			& 0.026(0.004)	 & 0.00(0.00)\\
    50					& 50  						& 1(0) & 1(0) 		& 1(0)				& 0.42(0.04)			& 0.015(0.002)	 & 0.00(0.00)\\
    50					& 100 						& 1(0) & 1(0) 		& 1(0)				& 0.41(0.05)			& 0.025(0.003)	 & 0.00(0.00)\\
    50					& 200 						& 1(0) & 1(0) 		& 1(0)				& 0.39(0.04)			& 0.05(0.02)	 & 0.00(0.00)\\
    100 				& 100	 					& 1(0) & 1(0) 		& 1(0)				& 0.55(0.04)			& 0.025(0.003)	 & 0.00(0.00)\\
    100 				& 200	 					& 1(0) & 1(0) 		& 1(0)				& 0.62(0.08)			& 0.0461(0.003)	 & 0.00(0.00)\\

 \bottomrule
	\end{tabularx}                                                                                                                                  
\end{table}


	\newpage
	\section{Discussion and Conclusions}	
	Reflecting on the material presented in this report, we find the following among the more significant aspects of CSP approximations by convex relaxations.
\begin{itemize}
\item The rounding schemes in \textit{How to Round Any CSP} and \textit{Optimal Algorithms ... For Every CSP?} are too slow to be implemented in practice.
\item The above rounding schemes are not only the only \textit{optimal} rounding schemes for Basic SDP, they are also the \textit{only} rounding schemes that apply to any k-CSP (regardless of whether or not that CSP is Unique).
\item There are generic rounding schemes for certain sub-classes of CSP's (particularly, any Unique CSP).
\end{itemize}
	
Graph coloring is an important benchmark CSP because its arity is 2, but the $\{\neq\}$ operator is non-unique in both arguments on domains of size $q > 2$. Interestingly, all approximation algorithms for graph coloring that we can find increase the number of colors, rather than failing to satisfy some adjacency constraints. 
	
	In addition to these theoretical limitations, we found that while the LP relaxation can be solved quickly, the time taken to solve the SDP relaxation was prohibitively slow for problems with thousands of variables. As a result, there are \textit{two} obstacles to using SDP relaxations of CSP's (1) the lack of rounding schemes for non-unique CSP's, and (2) the speed of SDP solvers themselves.
	
	Addressing the second problem requires substantial expertise in the implementation of convex optimization algorithms, but the first can potentially be handled in a variety of ways. Using the geometric interpretation of the columns in $Y : Y^\intercal Y = \Sigma^*$, it would be reasonable to classify these vectors according to cosine similarity. For CSP's which are invariant under a permutation of elements in the domain (including graph coloring), a cosine similarity classification is sufficient to characterize a rounding scheme. 
	
To use a heuristic approach with a more rigorous foundation, Prasad Raghavendra commented\footnote{in a private communication} that the algorithm proposed in \textit{How to Round Any CSP} could be implemented in spirit simply by projecting SDP vectors onto a very small constant number of directions (say, 3) rather than onto $1/\epsilon^2$ directions.

\iffalse
Note the following problem specific specific LP relaxations \cite{GoeWil94,Asa97,Yan94}.
\begin{itemize}
\item this thing
\item this thing as well
\item also this thing
\end{itemize}
k-SAT
graph coloring
\fi
	\bibliographystyle{abbrv}
	\bibliography{references/references}
	% appendix
	\section*{Appendix}
\subsection*{Notation in Raghavendra's Papers}
Unfamiliar and complex notation can make reading either of the original works (\textit{Optimal Algorithms .. For Every CSP?}, or \textit{How to Round Any CSP.}) difficult. In this section, we consolidate the notation used in these works to aid future readers.

\subsubsection*{Notation in \textit{Optimal Algorithms ... For Every CSP?} (2008)}

\subsubsection*{Notation in \textit{How to Round Any CSP} (2009)}	
	\section{Source Code}
\lstinputlisting[language=Matlab]{./code/CSP.m}
\newpage
\lstinputlisting[language=Matlab]{./code/Constraint.m}
\newpage
\lstinputlisting[language=Matlab]{./code/lprelax.m}
\newpage
\lstinputlisting[language=Matlab]{./code/lpExperiments.m}
\newpage
\lstinputlisting[language=Matlab]{./code/constructAndSolveSDP.m}
\newpage
\lstinputlisting[language=Matlab]{./code/gwSDP.m}
\newpage
\lstinputlisting[language=Matlab]{./code/gwRound.m}
\newpage
\lstinputlisting[language=Matlab]{./code/maxcutBipartite.m}
\newpage
\lstinputlisting[language=Matlab]{./code/maxcutExperiments.m}
\newpage
\lstinputlisting[language=Matlab]{./code/satGeneration.m}
\newpage
\lstinputlisting[language=Matlab]{./code/satRoundingLP.m}
\newpage
\lstinputlisting[language=Matlab]{./code/UGrounding.m}

	
\end{document}

